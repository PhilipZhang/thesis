% chap07.tex

\chapter{建模还原度度量}
\label{sec:qualityevaluation}
对于一个通过建模获得的树木模型,如果没有一个客观的量化评价指标,就无法从客观的角度
反馈树木模型的还原度和各个步骤算法的可行性。对于本文的基于图像序列的树木建模方法,
建模的输入是在自然环境下拍摄的树木图片序列,输出是三维的骨架模型。因此,判断三维模型
和投影照片的相似程度是评价建模质量的核心。然而,大多数基于图像的树木建模论文\cite{quanlong,
tanping,lichuan,tanping2,liu}只给出了输入图片和建模结果在少量角度的渲染效果,试图让
观察者从肉眼观察其相似度。但是这种方法是主观的,因观察者的不同可能会有不同的评价结果,
这显然不是一个好的评价方法。

为了客观、量化地评价基于图像序列的建模质量,本文提出了一套完整的评价方法。然而,仅仅凭借照片
无法完全表达出其所在环境的信息,比如环境光照,因为遮挡而产生的阴影信息等,因此本文的评价
方法将不针对模型的纹理和颜色信息,仅仅对模型的几何信息和照片中的几何信息的匹配程度进行量化
分析。

设树木模型$M$由$n$张从不同角度拍摄的同一棵树的图片序列$I_1\sim I_n$,经过基于图像的三维重建,
骨架抽取的方法进行建模所获得。那么模型$M$的建模还原度$\mathbb{Q}$定义如下:\\
\begin{definition}
	\[\mathbb{Q}=\mathbf{I}\cdot\mathbf{R_{3d}}\cdot\mathbf{R_s}\]
\end{definition}

建模还原度$\mathbb{Q}$的取值范围为$[0,1]$,0表示没有还原出任何树木几何信息,1表示准确还原出
整棵树木的几何信息。

本文将建模还原度$\mathbb{Q}$考虑由3个部分组成,为此也引入了三个新的概念:图片序列信息量$\mathbf{I}$,
三维重建还原度$\mathbf{R_{3d}}$,以及骨架抽取还原度$\mathbf{R_s}$。这三个分量的取值范围都为$[0,1]$,它们
的乘积即为总的建模还原度$\mathbb{Q}$。后续小节会详述这三个分量。

\section{图像序列信息量}
当实地对树木进行多角度拍摄时,拍摄者将基于不同的水平角度对树木进行全方位的拍摄,以便将整棵树的
信息尽可能多的携带进图像序列中。然而,从客观上来看,怎么样的图片序列才更加完整的表达了整棵树的
几何特征?为了从客观和量化的角度给出树木图像序列所携带的树木信息的多少,本文引入了图像序列信息量的概念。

那么,到底怎样的图片序列携带的信息量更大呢?从拍摄过程分析,如果想要得到一棵完整的树木信息,那么
需要绕着一棵树一圈进行密集地拍摄。这里的一圈,用数学化的表示,就是$360^\circ$,如果只是绕半圈进行
拍摄,那么所得到的图片序列表达的树木信息必定是不完整的,所以角度对信息量有着很大的贡献。另一方面,
如果每隔$60^\circ$进行一次拍摄,和每隔$30^\circ$进行一次拍摄,在它们都绕圈拍摄的前提下,后者的图片序列所
含信息量必然更大。再进一步思考,如果我隔$360^\circ,180^\circ,90^\circ,..., 1^\circ$进行拍摄呢?那么后一次拍摄所得的图
片序列相比前一次的图片序列信息量的增长是相同的吗?答案是否定的,因为当图片很少时,三维重建的结果也不好,
这时增加图片数量是能够很大程度上改观三维重建的重量的,因此此时的信息量增长速度快。
但是在拍摄已经比较密集的情况下,后一次拍摄所增加的信息量必定只是一些细节的信息,所以,信息量增长的速率应该变小。
并且一个信息量大的图片序列应该满足以下三个要求:\\
\begin{itemize}
	\item \textbf{图片数量多}: 图片数量多也就意味着拍摄角度多,因为一张图片代表着一个角度。
	\item \textbf{角度跨度大}: 跨度大指需要对树木进行全方位的拍摄。
	\item \textbf{角度分布均匀}: 若图片只是密集的集中在一个角度区间,就算图片再多,也无法完整地表达整
								  棵树的信息,所以若在角度多和跨度大的情况下还满足分布均匀,那么就能很完整
								  地携带树木的信息。
\end{itemize}

由于从平面的2D图像很难得到其空间角度拍摄情况,因此在这里我们简化其定义,将关注点放在图片数量上来,对于
图片跨度和角度的均匀分布,我们默认拍摄者在拍摄过程中采用均匀的角度偏差来进行$360^\circ$的拍摄。

根据以上的分析,本文给出了图像序列信息量的数学定义如下:\\
\begin{definition}
	\[ \mathbf{I}=1-\left[\frac{a}{b}\right]^n \]
\end{definition}

其中,图片序列信息量$\mathbf{I}$的取值范围为$[0,1]$。当$\mathbf{I}=0$时表示图片序列并不包含树木信息,
当$\mathbf{I}=1$时表示图片序列能完全表达空间树木的几何信息。$a,b$都是正数且$a<b$,具体数值需要对不同树木
进行实验之后才能得到。尽管$a$和$b$因树木特点不同而不同,但是它始终满足前文提出的信息量增长速度的特点,
即先快后慢。

\section{三维重建还原度}
对于一个给定的图片序列,所用三维重建方法所得到的点云模型的与实际的树木在几何形状上的相似度如何,由三维重建
还原度$\mathbf{R_{3d}}$来定义。注意,实际树木的几何信息被记录在输入的图像序列中,所以想要计算点云模型和实际树木
的相似度,就需要对点云模型和图片序列进行比较。然而对于三维的点云信息和二维的图片信息,无法进行直接地比较。一个
比较直观的想法,是对三维的点云进行投影,投影的角度由三维重建过程中的照相机几何标定步骤给出。

由于不考虑模型纹理和颜色信息,在空间点被投影到平面以后,只关注其是否在对应角度图片的树木轮廓内。所以对输入的树木
图片序列,需要先获得其轮廓图,并将其转化为二值图像。树木上的点值为1,而树木外的点值为0。对于每一个点云模型中的点,
按对应角度投影,获得其在对应图片上的坐标值,并且在其二值图像上确定其值,若为1,则表明匹配成功,否则匹配失败。最后
统计出匹配成功的总的比例,作为三维重建的还原度。

根据以上分析,本文给出三维重建还原度的数学定义式:\\
\begin{definition}
	\[ \mathbf{R_{3d}}=\frac{1}{n}\sum_{i=1}^n \frac{P_i}{P_i+O_i}\]
\end{definition}

上式中的$n$表示图像的数量,$P_i$表示点云模型投影到第$i$张图片上在树木轮廓中的点的数量,$O_i$表示点云模型投影到第
$i$张图像上在树木轮廓外的点的数量,因此$P_i+O_i$自然就表示点云模型中点的总数量。$\frac{P_i}{P_i+O_i}$表示点云投影到
第i张图片上的击中率。最后对每张图像的击中率求平均,作为总的三维重建的还原度。其值区间为$[0,1]$。

\section{骨架抽取还原度}
骨架抽取是基于三维点云模型进行的,因此计算骨架抽取的还原度的输入是重建出的点云模型和抽取出的骨架模型。由于点云模型是
三维的点的集合,而抽取出的骨架模型却是一个记录着树形结构的逻辑信息,它们无法进行直接的比较。本文采取的做法是将骨架的
树形逻辑信息用圆台和球进行堆叠从而将其转化为三维的表示。

具体的做法是对骨架中的每个节点,根据其半径构造出一个球体。然后对于每个父子关系,用一个圆台来表示其枝干,圆台的底半径
等于父节点的半径,圆台的顶半径等于子节点的半径。然后对于每个点云模型中的点,用数学公式判断其是否存在于骨架的三维表示中
的球体或圆台中,如果存在,则表示匹配成功,否则表示匹配失败。最后用成功点数与总点数的比值来表示骨架抽取的还原度。定义如下:\\
\begin{definition}
	\[ \mathbf{R_s}=\frac{S}{N} \]
\end{definition}

其中$S$表示匹配成功的点数,而$N$表示点云模型的总点数。$\mathbf{R_s}$的值区间为$[0,1]$。

注意,若用经过枝干合并轻量化处理的骨架进行骨架抽取还原度计算,其值必定会比直接从点云中抽取出来的模型要小,因为模型经过
简化后,与原点云模型的匹配度也必将降低。本文的目标只是尽可能在还原度降低不多的情况下,对骨架进行尽可能多的轻量化。

\section{建模还原度计算}
将图片序列信息量$\mathbf{I}$、三维重建还原度$\mathbf{R_{3d}}$和骨架抽取还原度$\mathbf{R_s}$代入建模还原度$\mathbb{Q}$
的定义式中,可以得到建模还原度的计算式:\\
\begin{equation}
	\mathbb{Q}= (1-\left[\frac{a}{b}\right]^n)\cdot \frac{1}{n}\sum_{i=1}^n \frac{P_i}{P_i+O_i} \cdot \frac{S}{N}
\end{equation}

\section{本章小节}
本章提出了基于图像树木轻量化建模的质量评价方法。首先提出了建模还原度的概念,它包含了三个子项:图像序列信息量、
三维重建还原度以及骨架抽取还原度。它们分别代表了图像序列对真实树木的信息携带量、点云模型和图像序列的匹配度、骨架模型与点云模型的匹配度。
本文对这三个子项的由来和计算方法都进行了阐述,并将它们融合给出了建模还原度的计算式。
