
\chapter{总结与展望}

\section{总结}
本文首先总结了国内外树木建模工作者的技术与方法,并分析了它们的优缺点。为了
得到真实感强的树木模型,本文从改进了传统三维重建中的匹配算法。使得PyrLK算法
适应了仿射变换和容错机制,从而成功地提高了三维重建的准确性和鲁棒性。然后进一步
对三维重建得到的点云模型体素化,并用基于三维体素泛洪和拟合的办法抽取出树木的
骨架拓扑结构。然后根据枝干骨架的纵向合并和横向合并,对树木骨架进行分级的轻量化。
在本文的最后,提出了一种基于图像序列树木建模的还原度计算方法,作为其建模的质量
评价的量化标准。

本文提出的树木建模方法先从准确三维重建方法入手,该方法可以获得极高还原度的点云。
并且基于该点云模型,进行符合树木生长特性的骨架抽取,得到的骨架还原度也极高。在
高还原度的骨架的基础上,本文提供了枝干合并的轻量化处理办法,因此可以对树木进行
分级别的建模,从精确的模型到极度轻量化的模型,该方法可以支持一系列级别跨度的应用。
无论是桌面要求精细树木模型的应用,还是嵌入式,Web等要求轻量化树木模型的应用,本文的
方法都可以很好的满足它们的要求。

\section{未来的工作}
本文方法的未来发展方向如下:\\

1. 本文采取的建模输入是图像序列,要让拍摄者密集地对图像进行拍摄,是很不方便的。因此
未来可以考虑将视频作为输入,这样围绕树一圈拍一个视频将会比密集拍照要方便的多。而且视频
中的相邻帧之间的密集程度也会比图像大,从而使得三维重建更加准确。

2. 本文使用三维重建的方法是基于树木图片进行点云重建,所重建的点云呈现表皮化。虽然经过
三维的点云修复,但是这种点云修复是带有人为主观性和二义性的,究其原因还是受缚于基于点云的操
作的复杂性。对于本身几何连续的树木来说,未来将尝试基于三维体素泛洪的三维重建,使得三重
重建出来的结果就是连续的体素,而不是空心的点云,以便于后续处理。

3. 本文只是做了单棵树木的轻量化工作,未来可以考虑将其扩展为基于单棵树木轻量化骨架形式而
随机扰动形成森林,以适应大规模树木轻量化建模的要求。
