
\chapter{基于图像的树木轻量化3D建模方法}
\label{cha:techroute}

\section{技术路线}
\label{sec:techroute}
本文提出了一套完整的基于图像的树木轻量化3D建模的方法。该方法首先以树木图片
序列作为输入,用经过改进的方法对树木进行三维重建,使三维重建得到的模型精确
度和完整性都得以提高。然后再用基于空间方向迭代和步长探索的方法抽取树木的骨
架,最终再基于用户交互对骨架进行改善与轻量化。

该技术路线旨在实现一个对建模设备和条件要求不高,适用于一般应用的方法。在方便
和简单的基础上,尽可能多的加入自动化,并结合少量用户交互,以实现高效、精确的
树木轻量化3D建模。

该方法的主要步骤如下:

\subsection{基于改进的PyrLK光流法的图像特征匹配}
\label{sec:match}
基于图像的树木建模第一步是三维重建,而三维重建的第一步则是特征点的匹配。
所谓的特征点匹配,是在多张图片中找到空间同一个点在其上的投影位置,从而为三维
重建的后续步骤提供数据支持。这里的特征点,本文选择了具有平移和旋转不变性的Harris\cite{harris}
角点,以便快速找出图片中的所有特征点。然后再结合改进的LK金字塔光流法\cite{Lucas81aniterative}
,对找到的特征点在一定的容错区间进行匹配,最终将匹配结果存储到匹配文件以供后续
使用。

\subsection{照相机的几何标定}
\label{sec:calibration}
特征匹配完成以后,本文使用了美国华盛顿大学西雅图分校Changchang Wu的可视化运动
恢复工具VisualSFM\cite{vsfm} 来完成照相机的几何标定。VisualSFM实现了GPU加速\cite{siftgpu}
和多核的捆集调整(Multicore Bundle Adjustment)\cite{mba}, 使得相机参数的恢复更加
快速和精确。在这个步骤本文用经过改进的PyrLK光流法的匹配结果替换VisualSFM中的
SIFT特征点匹配文件,再一次地改进了相机参数恢复的准确度和可信度。

\subsection{基于体素泛洪与空间反向投影的三维重建}
\label{sec:3drec}

\subsection{基于多方向迭代与步长探索的三维树木骨架抽取}
\label{sec:treesklextract}
在完成了三维重建之后,将会得到一个比较完整的树木空间体素模型。本文根据该体素的
空间分布,并结合树木自底向上的自然生长规律和分形的逻辑结构特征,在阈值范围内,
不断增加步长来扩大邻域范围,并且向多个子方向进行扩展来寻找可能的分支。同时在迭
代过程中及时剔除已经形成枝干的体素,来加速迭代算法的完成。最终获取到的骨架信息
是含有父子关系的节点信息,相比起3DSMAX等手工工具导出的面片模型,这种逻辑结构的
模型大大的减小了其尺寸,但是由于逻辑结构并没有多少丢失,所以极具真实感。并且这种
结构更便于后续的处理和进一步轻量化。

\subsection{基于用户交互的模型改善与轻量化}
\label{sec:userinteraction}
由前面方法所得到的树木三维骨架虽然已经是含有父子信息的树木逻辑结构,但是由于前面
的步骤都是自动化生成,所得到的结果不可能100\%地保证符合具体应用的需求。并且前面的
骨架信息虽然比起用面片来表示树木模型已经大大的轻量化了,根据实际应用的需要,

\subsection{建模质量评估}
\label{sec:qualityevaluation}

\section{技术路线图}
\label{sec:techrouteimg}

