
%%% Local Variables:
%%% mode: latex
%%% TeX-master: t
%%% End:
%\secretlevel{保密} \secretyear{2}

\ctitle{基于图像的轻量化3D树木建模方法}

% 按照申请工学学位设计。如有其它需要,请修改相应文字。
\makeatletter
  \iftongji@doctor
    \cdegree{工学博士}
  \else
    \iftongji@master
      \cdegree{毕业设计(论文)}
    \fi
  \fi

\makeatother

\cdepartment{软件学院}

\cmajorfirst{软件工程}

\cmajorsecond{软件工程}

\cauthor{张德嘉}

\snumber{092792}

\csupervisor{贾金原}

% 如果没有副指导老师或者联合指导老师,把各自{}中内容留空即可。

\cassosupervisor{}

\ccosupervisor{}

% 定义中英文摘要和关键字
\begin{cabstract}
树木建模一直是计算机图形学中一个极具挑战并且非常重要的研究课题。随着目前WebVR、WebGame、WebGIS等基于Web的应用
发展迅速,为了适应网络的传输以及用户日益增长的对图形效果的追求,如何使树木建模轻量化而富有真实感就变得尤为重要。
传统的3DSMAX、Maya等建模工具不仅耗费人力物力,而且输出的面片模型也体积庞大,不适合应用到Web领域。而诸如L-System
等基于规则的树木建模又由于其规则性而使树木模型缺失了真实感,这又不能满足用户对效果的需求。怎样在真实感和轻量化之间
进行权衡的问题亟待解决。

为了解决这个矛盾,本课题提出了一套高效、低成本,的分级轻量化树木建模方法。这里的分级轻量化体现为其对应用的适应性。
即可以基于不同应用对轻量化的不同要求,在尽可能保证真实感的前提下进行轻量化,以产生最终符合要求的模型尺寸。这种可分级
的轻量化树木建模方法还可以进一步被扩展为自动适应网络带宽条件或用户硬件条件而自动产生最匹配的轻量化树木模型的方法。

为了实现高效的分级轻量化建模方法,本文首先将PyrLK光流法进行基于仿射变换和反向追踪的改进,并且将其运用到三维重建
的特征点匹配步骤中,以提高树木特征点的匹配率和稳定性。然后进行GPU加速的三维重建以得到高精度点云模型。接着本文
运用三维体素泛洪和最小二乘线性拟合的方法对树木骨架和半径信息进行抽取,以适应树木生长规律的方法抽取出了准确的骨架。
然后本文提出了根据应用对轻量化的需求等级,对骨架进行纵向和横向的合并,以减小骨架的尺寸来实现轻量化,从而更好地适应
面向网络的应用的需求。最后本文还给出了一套完整的基于图像树木建模的质量评价,提出了还原度的概念来客观、量化地评价建模出
的模型的还原度以及在轻量化过程中质量的丢失。
\end{cabstract}

\ckeywords{基于图像建模, 树木建模, 轻量化建模, 三维重建, 骨架抽取}

\begin{eabstract}
Tree modeling has long been a challenging subject in computer graphics. As the Web-oriented applications(WebVR, WebGame,
WebGIS, etc) develop rapidly and the persuit of graphics effect increases quickly, the lightweight and realism of tree modeling are badly
needed nowadays. The traditional 3d modeling tools such as 3DSMAX and Maya are not only time and labour consuming, but it
takes a large model size, which are not practical to Web apps. The rule-based modeling such as L-System can solve the size
problem, but its output lacks realism, which is not tolerated by users. So the balance between realism and lightweight
is a real problem which are eagerly demanded to solve.

In order to solve this problem, a high-efficiency, low-cost and lightweight-classified tree modeling method is proposed.
Here the lightweight-classified means it can produce different lightweight levels of tree models. And to implement this 
goal, this method will reduce model size on the premise of not losing much realism. This method can also be applied furthur
to automatically adapt to the bandwidth and hardware conditions of the client side.

For implementing the lightweight method, we first improve the traditional PyrLK optical flow method to support affine transformation
and backward feature tracking, which can furthur be applied to do feature matching in gpu accelerated 3D reconstruction and 
improve the match ratio. Then we use flooding algorithm in 3D voxel model and least squares method to discover the tree skeleton and
its radius information. According to the lightweight level the applications require, we reduce the model size by merging the
branches vertically and horizontally respectively. At last we propose a modeling quality evaluation method, which will objectively and 
quantizedly evaluate the restore degree of the tree model.
\end{eabstract}

\ekeywords{Image-based Modeling, Tree Modeling, Light-weight Modeling, PyrLK Optical Flow, Skeleton Extraction}
