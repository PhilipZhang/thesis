
%%% Local Variables:
%%% mode: latex
%%% TeX-master: t
%%% End:



\chapter{引言}
\label{cha:intro}
\section{背景介绍}
\label{sec:background}
在互联网飞速发展的今天,网络应用已经延伸到生活的方方面面。微博、人人网、
在线购物、在线音乐等已经成为当今人们生活的一部分。面向Web的虚拟现实应用
如WebVR、WebGame、WebGIS等也必然将成为虚拟现实发展的重要方向。树木作为自然
界常见的事物,在各种虚拟现实的场景中出现的频率很高。然而树木形态各异,结构
复杂,给3D建模带来了很大的难度。通常单棵树的数据量已经不小,对于构建一个
树木的聚集场景(如森林)就更加庞大,这容易使得场景负荷变大而产生延迟。因此,
树木建模的质量和效率将直接决定面向Web的虚拟现实应用的成败。

目前的树木的3D建模,主要是通过专业的3D建模工具(3DSMAX、Maya等)进行手工建模。
这种建模方法对建模人员的要求较高,并且需要的时间长。而且这种方法通常最终生成
的是面片信息,要表达一棵形态复杂的树木需要大量的顶点信息,导致最终生成的模型
体积较大,对于需要大批树木的场景,负荷就会变得更大。

目前树木的轻量化建模,从最简单的基于分形,广告牌技术的建模到稍微复杂的基于
规则的建模,都存在一个共同的问题,就是在轻量化的同时,很大程度上舍弃了模型
的真实感和树木本身的形态特征。随着当今应用对真实度要求的升高,这类轻量化的
建模方法已经不能完全满足需求。真实感与轻量化之间的权衡也成为了当今应用需要
考虑的一个重要因素。

本课题基于以上的考虑,从基于图片对树木结构进行完整的恢复,到面向应用需要对
真实感与轻量化进行人工控制,到最后模型重建质量的评估,给出了一套完整的解决
方案。

\section{课题的主要工作}
\label{sec:objective}
本课题的主要工作有:

1. 对PyrLK光流法进行改进,并将其运用于三维重建算法中的特征点匹配步骤,使树木
重建的模型更加准确和精细。

2. 提出了基于三维体素泛洪和线性拟合的树木骨架抽取方法。该方法区别于传统的3D瘦化
骨架抽取方法,它只适用于具有分形结构的3D骨架,所以更能够准确的抽取出树木的骨架。

3. 提出了基于用户交互对树木模型进行完善和轻量化,让最终的应用来决定其所需的树木
模型,避免了主观的一味轻量化或一味追求真实感而带来的需求矛盾,将模型的成型延迟
至具体的应用。

4. 提出了基于图像的树木重建质量评估方法,对于建模质量和轻量化过程中真实感的下降
程度给出了函数化和量化的评价依据。
