
%%% Local Variables:
%%% mode: latex
%%% TeX-master: t
%%% End:



\chapter{树木建模的研究综述}
\label{cha:methodsandtechniques}

树木的建模和造型技术,是计算机图形领域颇具挑战性的研究方向之一。自上世纪六十年代起,
大批国内外的研究工作者利用各种不同的方法和技术来构建树木的形态,大大地推进了树木建模
技术的发展。目前,在树木的建模领域,主要存在两种不同的类别:虚拟树木建模和对真实世界
的树木进行重建。

\section{虚拟树木建模技术}
虚拟树木建模主要是指所建模的树木对象并不是直接从现实生活中获取,而是根据一定的规律
或生长机理模拟化地对树木进行建模,树木的结构等都是通过过程化的方法所生成,而非从点云
中抽取。

一个虚拟树木建模的经典方法是由Lindenmayer于1968年提出的L-System\cite{lsystem},它是一个“字符串重写
系统”,后在90年代初,Prusinkiewicz与Lindenmayer一起将L-System规则系统用于描述树木的
生长过程\cite{beauty,devmodels}。它用语法表达了植物的生长规则,加入了分枝角度、长度等
信息,以便于植物的表达与生长。在此后以L-System为基础的研究工作中,部分研究人员用若干
几何模型构建出植物的枝干,并且引入不同的参数来表示植物的生长。另一部分研究人员使用分形
的方法来进行树木建模。近几年来,L-System的方法常常以用户勾画为引导,让用户在简便操作的
前提下设定L-System的参数,从而构建出树木模型,Okabe在2006年,Anastacio在2009年\cite{sketch}都在这方面
作出了贡献。2010年,Hongchun Qu使用BHA自动机和马尔科夫方法自动化地从输入图像中提取了L-System
规则,在自动化提取规则方面迈出了第一步。

虚拟树木建模的另一个研究领域是AMAP系统\cite{amap}。该系统通过观察植物的结构,对植物形式和结构
获得定性地理解,然后定量的测量植物形态的数据。植物的生长有一定的随机性,通过概率分
布和应用理论来表达随机过程。系统依靠强大的实地数据采集和分析模块,将植物的各项测量数据整合
到植物数据库,植物的拓扑结构演化由马尔可夫过程进行分析获得。再通过模式识别方法分析数据中提
取生长规则的类型来构造植物的几何形状,应用蒙特卡罗的方法仿真模拟植物的模型,再应用几何的方
法来表达其形成规律,并由此制作模型参数表,最后在场景中生成植物的图形。

\section{现实树木重建技术}
现实树木重建是指从现实生活中通过照片,视频或者三维扫描,来获得树木的实际数据,通过一系列重建
的方法来对树木的几何,物理信息进行复原的过程。

基于图像的树木建模技术,是以在现实中拍摄的树木的图片作为为输入,然后根据图片附带的树木信息
重建出树木结构的技术。2004年,Reche-Martinez以空间体素的形式重建出了照片中的树\cite{reche}。
2007年,Neubert不仅近似的生成了空间体素,而且还进一步以3D粒子流的方法来模拟生成了细枝和枝
干\cite{neubert}。该方法将树木上的点看作吸引子,以物理的吸引力等概念重建出了树木的信息,十分新颖。
图片同样可以用来抽取L-System规则,对于树冠密集的树木,Shlyakhter在2001年首次从图片中抽取出了
可见部分的规则,然后运用L-System进行处理。对于树叶占据大部分区域的树木图像,香港科技大学学者
权龙在2006年,用交互式勾画辅以稀疏3D重建的方法,很好的恢复了树木的信息\cite{quanlong}。而对于
树干信息占据大部分区域的树木图像,谭平又分别在2007年和2008年,用自动化L-System和用户交互L-System
的方法完成了树木的重建\cite{tanping,tanping2}。2010年,Luis D.Lopez等人提出了一种从稀疏图像
序列重建无叶树木的方法。2011年,Chuan Li等人提出了一种从树木视频输入重建树木模型的方法\cite{lichuan}。
这两种方法均先从图像中获取2D树木骨架,然后对2D树木骨架序列进行三维重建获得三维骨架。由于树木图像
本身存在遮挡导致2D树木骨架无法准确抽取,从而影响3D树木骨架的生成,因此这两种方法还原度都比较低。

基于激光扫描仪的体素化模型生成技术,是一种更直接地现实树木重建技术。它利用激光的单色性好、方向
性强、能量高、光束窄等特点,直接对树木进行激光扫描,从而得到非常密集的点云数据。大多数基于激光
扫描仪的方法都将重点放在了恢复代表枝干的骨架上,因为扫描出的叶子的点云含有太多噪声,以致无法准确
重建其信息。1999年,Lazarus和Verroust用生成树的边长来聚合点云,从而获取骨架\cite{verroust}。
Bucksch在2008年和2009年将点云分块到八叉树表示的格子,然后用相邻格子间的连线来模拟骨架的曲线\cite{bucksch}。
2007年Xu用启发式的方法来从扫描的点云中重建树木的主干,然后再用人工合成的办法在其上添加细枝和叶子\cite{xu}。
C\^ote在2009年同样用人工合成的方式去构建细枝和树叶,但是他对光照散步的合成是通过在扫描的时候进行
采样\cite{cote}。

\section{现有方法的缺陷}

如表\ref{tab:short}所示,基于L-System的方法虽然轻量化,但是其试图用少量的规则来刻画自然中本就不规则生长的复杂的树木,
导致了真实感的缺乏,同时在一个复杂结构中抽取L-System本身也是带有人为主观性和二义性的。基于AMAP系统的
树木建模,虽然其建模的还原度较高,但是它需要大量的数据采集和专业的植物学知识,这对于一般性的应用来说
显得超负荷。对于目前的基于图像的建模技术,如何准确的进行三维重建和骨架恢复仍然是一个难点。而基于激光
扫描仪的树木建模,设备的价格又太高昂,而且对于骨架的抽取仍有一定的二义性存在。
\begin{table}[H]
	\centering
	\caption{现有树木建模方法的缺点}
	\label{tab:short}
	\begin{tabular}{|l|l|}
		\hline
		方法 & 缺点\\
		\hline
		L-System & 缺乏真实感,带有主观性和二义性\\
		\hline
		AMAP系统树木建模方法 & 采集大批植物数据和专业知识太过耗费时间和精力\\
		\hline
		现有基于图像的树木建模方法 & 三维重建和骨架恢复不够准确\\
		\hline
		基于激光扫描仪的树木建模 & 设备价格过高,不适用于一般应用\\
		\hline
	\end{tabular}
\end{table}

\section{本实验室的先期工作及不足之处}

同济大学图形图像实验室基于此做了一些先备工作\cite{WengHao,ZhangChen,SunRuoXi},对基于图像的树木轻量化建模提出了建设性的思想和方法。其采用
PyrLK光流法,改善了三维重建的特征点匹配步骤。并且运用基于空间反向投影的方法,重建树木点云。之后再利用
邻域探索点云分布的方法对树木骨架进行抽取,并通过L-System方法对骨架进行了规则抽取,以达到轻量化。虽然该方法
对以往的一些方法进行了完善和改进,但是仍然存在以下不足:\\
\begin{itemize}
	\item \textbf{特征点匹配缺乏准确性和鲁棒性} 虽然PyrLK光流法代替SIFT特征点匹配,从一定程度上改善了树木的
		特征点匹配效果。但是由于无法捕捉特征点的旋转变换,使得其特征点匹配的丢失情况严重。而且由于匹配缺乏一
		个验证正确性的机制,导致出现了过多的错误匹配。
	\item \textbf{骨架抽取不准确、不完整} 由于骨架抽取步骤中邻域的扩展运用增加边长的方法,在寻找一个分支的时候,容易
		受到其他分支影响,并且由于增加边长的方法是“暴力”的,导致效率很低。而且该骨架抽取方法没有给出合理的半径
		和长度的估算公式,导致根本无法还原出完整树木骨架信息。
	\item \textbf{轻量化效果欠佳} 对于自然中真实感极强的树木,利用L-System并不能很好的迭代地搜索出相同的规则,这
		是因为树木的结构本就是多变和复杂的。运用参数化L-System的方法将产生大量的规则字符串,而导致没有从根本上解决轻量化
		的需求。
	\item \textbf{缺乏有效的交互手段} 对于形态结构复杂的树木,单纯依靠计算机重建实在无法完成高质量(100\%)树木模型的
		重建。因为自动化算法运用到树木上时无可避免地会产生主观性和二义性,因此无可避免地需要少量的人工指引。而目前有效
		的交互手段仍然缺乏。
\end{itemize}

为了解决这些问题,弥补现有方法的局限,本课题结合现有方法的优点,改进其不足,提出了一整套基于多张图像
的树木轻量化建模的解决方案。该方法较好的综合了真实感与轻量化,产生了很好的效果。

\section{本章小节}
国内外学者们多年来通过各种不同的技术和思想对树木建模的发展作出了卓越共享,也为后续工作者的研究
奠定了坚实的基础。本章首先简要介绍了两大类树木建模的领域,即虚拟树木建模与现实树木重建。然后
通过总结这两大领域中的已有方法,从规则生成,植物学领域,三维重建,骨架抽取等角度思考了树木建模
问题。在同济大学图形图像实验室的已有工作上,本文进行进一步的创新和完善,并提出了一套兼具理论性和
实践性的基于图像树木轻量化建模方法。
