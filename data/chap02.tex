
%%% Local Variables:
%%% mode: latex
%%% TeX-master: t
%%% End:



\chapter{研究综述与技术简介}
\label{cha:methodsandtechniques}

\section{树木建模方法综述}
\label{sec:treemodelingmethods}
树木的建模和造型技术,是计算机图形领域颇具挑战性的研究方向之一。自上世纪六十年代起,
大批国内外的研究工作者利用各种不同的方法和技术来构建树木的形态,大大地推进了树木建模
技术的发展。目前,在树木的建模领域,主要存在两种不同的类别:虚拟树木建模和对真实世界
的树木进行重建。

\subsection{虚拟树木建模技术}
虚拟树木建模主要是指所建模的树木对象并不是直接从现实生活中获取,而是根据一定的规律
或生长机理模拟化地对树木进行建模,树木的结构等都是通过过程化的方法所生成,而非从点云
中抽取。

一个虚拟树木建模的经典方法是由Lindenmayer于1968年提出的L-System\cite{lsystem},它是一个“字符串重写
系统”,后在90年代初,Prusinkiewicz与Lindenmayer一起将L-System规则系统用于描述树木的
生长过程\cite{beauty,devmodels}。它用语法表达了植物的生长规则,加入了分枝角度、长度等
信息,以便于植物的表达与生长。在此后以L-System为基础的研究工作中,部分研究人员用若干
几何模型构建出植物的枝干,并且引入不同的参数来表示植物的生长。另一部分研究人员使用分形
的方法来进行树木建模。近几年来,L-System的方法常常以用户勾画为引导,让用户在简便操作的
前提下设定L-System的参数,从而构建出树木模型,Okabe在2006年,Anastacio在2009年\cite{sketch}都在这方面
作出了贡献。2010年,Hongchun Qu使用BHA自动机和马尔科夫方法自动化地从输入图像中提取了L-System
规则,在自动化提取规则方面迈出了第一步。

虚拟树木建模的另一个研究领域是AMAP系统\cite{amap}。该系统通过观察植物的结构,对植物形式和结构
获得定性地理解,然后定量的测量植物形态的数据。植物的生长有一定的随机性,通过概率分
布和应用理论来表达随机过程。系统依靠强大的实地数据采集和分析模块,将植物的各项测量数据整合
到植物数据库,植物的拓扑结构演化由马尔可夫过程进行分析获得。再通过模式识别方法分析数据中提
取生长规则的类型来构造植物的几何形状,应用蒙特卡罗的方法仿真模拟植物的模型,再应用几何的方
法来表达其形成规律,并由此制作模型参数表,最后在场景中生成植物的图形。

\subsection{现实树木重建技术}
现实树木重建是指从现实生活中通过照片,视频或者三维扫描,来获得树木的实际数据,通过一系列重建
的方法来对树木的几何,物理信息进行复原的过程。

基于图像的树木建模技术,是以在现实中拍摄的树木的图片作为为输入,然后根据图片附带的树木信息
重建出树木结构的技术。2004年,Reche-Martinez以空间体素的形式重建出了照片中的树\cite{reche}。
2007年,Neubert不仅近似的生成了空间体素,而且还进一步以3D粒子流的方法来模拟生成了细枝和枝
干\cite{neubert}。该方法将树木上的点看作吸引子,以物理的吸引力等概念重建出了树木的信息,十分新颖。
图片同样可以用来抽取L-System规则,对于树冠密集的树木,Shlyakhter在2001年首次从图片中抽取出了
可见部分的规则,然后运用L-System进行处理。对于树叶占据大部分区域的树木图像,香港科技大学学者
权龙在2006年,用交互式勾画辅以稀疏3D重建的方法,很好的恢复了树木的信息\cite{quanlong}。而对于
树干信息占据大部分区域的树木图像,谭平又分别在2007年和2008年,用自动化L-System和用户交互L-System
的方法完成了树木的重建\cite{tanping,tanping2}。2010年,Luis D.Lopez等人提出了一种从稀疏图像
序列重建无叶树木的方法。2011年,Chuan Li等人提出了一种从树木视频输入重建树木模型的方法\cite{lichuan}。
这两种方法均先从图像中获取2D树木骨架,然后对2D树木骨架序列进行三维重建获得三维骨架。由于树木图像
本身存在遮挡导致2D树木骨架无法准确抽取,从而影响3D树木骨架的生成,因此这两种方法还原度都比较低。

基于激光扫描仪的体素化模型生成技术,是一种更直接地现实树木重建技术。它利用激光的单色性好、方向
性强、能量高、光束窄等特点,直接对树木进行激光扫描,从而得到非常密集的点云数据。大多数基于激光
扫描仪的方法都将重点放在了恢复代表枝干的骨架上,因为扫描出的叶子的点云含有太多噪声,以致无法准确
重建其信息。1999年,Lazarus和Verroust用生成树的边长来聚合点云,从而获取骨架\cite{verroust}。
Bucksch在2008年和2009年将点云分块到八叉树表示的格子,然后用相邻格子间的连线来模拟骨架的曲线\cite{bucksch}。
2007年Xu用启发式的方法来从扫描的点云中重建树木的主干,然后再用人工合成的办法在其上添加细枝和叶子\cite{xu}。
C\^ote在2009年同样用人工合成的方式去构建细枝和树叶,但是他对光照散步的合成是通过在扫描的时候进行
采样\cite{cote}。

\section{基于多张图像的三维重建技术简介}
\label{sec:mulimg3drecmethods}

\section{本章小节}
