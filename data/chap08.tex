\chapter{基于用户交互的模型改善}
通过自动化算法生成的模型难免含有算法设计者的主观想法在其中,比如树木何时应该分支,何时
应该合并等等。未免过于主观地决定了模型的最终成型,本文提倡在自动化生成模型以后,应该将模型
的建立延迟到应用程序,或者使用该模型的最终用户。这样可以使得本文所建立的一系列方法适用于
更广的情形。将自动化算法与人工交互联合使用,可以大大地提高最终模型与具体需求之间的耦合度。

本文根据此需求,开发完成了一个树木模型用户交互平台,可以用于对本文的基于图像所建立的树木
模型进行进一步地编辑和完善。该平台的具体功能见用例图\ref{fig:usecase}。

对于每个用例,本文用图片加文字加以说明:\\

