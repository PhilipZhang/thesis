%%% Local Variables:
%%% mode: latex
%%% TeX-master: t
%%% End:

\documentclass[master, openany, oneside]{tongjithesis}
%开始用了openright,但这样即每章都是奇数页开始,但实际上,只需要摘要、目录、英文摘要、正文第一章、
%参考文献、附录、个人简历都从奇数页开始即可,正文内部章节可以从偶数页开始,因此用\cleardoublepage命令实现

% \documentclass[%
%   master|doctor, % mandatory option
%   xetex|pdftex|dvips|dvipdfm, % optional
%   secret,
%   openany|openright,
%   arialtoc,arialtitle]{tongjithesis}

% 所有其它可能用到的包都统一放到这里了,可以根据自己的实际添加或者删除。
\usepackage{tongjiutils}
\usepackage[top=1.3in,bottom=1.15in,left=1.25in,right=1.25in]{geometry}%调节每一页的页面大小,
\usepackage[figuresright]{rotating}
\usepackage{graphicx}
\usepackage{titlesec}
\usepackage{algorithmicx}
\usepackage{algpseudocode}
\usepackage{etex}
\usepackage[perpage,symbol*,bottom]{footmisc} %使用了\raggedbottom后仍然保持脚注在页面底部,且脚注在每一页都从1开始,
%符合规范,脚注形式为①,另外,此命令与下面命令还可以使得模板原先的自动填充的空白得到很好的控制,如果没有这个命令,
%章节标题与正文间的空白很难控制,特别是有图、表与公式的时候
\raggedbottom
\usepackage{epstopdf}
\usepackage{fancyhdr}
\usepackage{verbatim}
\usepackage{CJKpunct}
\usepackage{setspace}
\usepackage{longtable}
\usepackage{url}
\usepackage{pifont}
\usepackage{algorithm}
\usepackage{ulem}
\usepackage{extarrows}
\usepackage{caption}
\setlength\parskip{0pt}
\frenchspacing
\widowpenalty=10000
%此命令貌似是用来修改图标与正文的间距的


% 你可以在这里修改配置文件中的定义,导言区可以使用中文。
\def\myname{同济人}

\begin{document}

% 定义所有的eps文件在 figures 子目录下
\graphicspath{{figures/}}


%%% 封面部分
\frontmatter

%%% Local Variables:
%%% mode: latex
%%% TeX-master: t
%%% End:
%\secretlevel{保密} \secretyear{2}

\ctitle{基于图像的轻量化3D树木建模方法}

% 按照申请工学学位设计。如有其它需要,请修改相应文字。
\makeatletter
  \iftongji@doctor
    \cdegree{工学博士}
  \else
    \iftongji@master
      \cdegree{毕业设计(论文)}
    \fi
  \fi

\makeatother

\cdepartment{软件学院}

\cmajorfirst{软件工程}

\cmajorsecond{软件工程}

\cauthor{张德嘉}

\snumber{092792}

\csupervisor{贾金原}

% 如果没有副指导老师或者联合指导老师,把各自{}中内容留空即可。

\cassosupervisor{}

\ccosupervisor{}

% 定义中英文摘要和关键字
\begin{cabstract}
树木建模一直是计算机图形学中一个极具挑战并且非常重要的研究课题。随着目前WebVR、WebGame、WebGIS等基于Web的应用
发展迅速,为了适应网络的传输以及用户日益增长的对图形效果的追求,如何使树木建模轻量化而富有真实感就变得尤为重要。
传统的3DSMAX、Maya等建模工具不仅耗费人力物力,而且输出的面片模型也体积庞大,不适合应用到Web领域。而诸如L-System
等基于规则的树木建模又由于其规则性而使树木模型缺失了真实感,这又不能满足用户对效果的需求。怎样在真实感和轻量化之间
进行权衡的问题亟待解决。

为了解决这个矛盾,本课题提出了一套高效、低成本,的分级轻量化树木建模方法。这里的分级轻量化体现为其对应用的适应性。
即可以基于不同应用对轻量化的不同要求,在尽可能保证真实感的前提下进行轻量化,以产生最终符合要求的模型尺寸。这种可分级
的轻量化树木建模方法还可以进一步被扩展为自动适应网络带宽条件或用户硬件条件而自动产生最匹配的轻量化树木模型的方法。

为了实现高效的分级轻量化建模方法,本文首先将PyrLK光流法进行基于仿射变换和反向追踪的改进,并且将其运用到三维重建
的特征点匹配步骤中,以提高树木特征点的匹配率和稳定性。然后进行GPU加速的三维重建以得到高精度点云模型。接着本文
运用三维体素泛洪和最小二乘线性拟合的方法对树木骨架和半径信息进行抽取,以适应树木生长规律的方法抽取出了准确的骨架。
然后本文提出了根据应用对轻量化的需求等级,对骨架进行纵向和横向的合并,以减小骨架的尺寸来实现轻量化,从而更好地适应
面向网络的应用的需求。最后本文还给出了一套完整的基于图像树木建模的质量评价,提出了还原度的概念来客观、量化地评价建模出
的模型的还原度以及在轻量化过程中质量的丢失。
\end{cabstract}

\ckeywords{基于图像建模, 树木建模, 轻量化建模, 三维重建, 骨架抽取}

\begin{eabstract}
Tree modeling has long been a challenging subject in computer graphics. As the Web-oriented applications(WebVR, WebGame,
WebGIS, etc) develop rapidly and the persuit of graphics effect increases quickly, the lightweight and realism of tree modeling are badly
needed nowadays. The traditional 3d modeling tools such as 3DSMAX and Maya are not only time and labour consuming, but it
takes a large model size, which are not practical to Web apps. The rule-based modeling such as L-System can solve the size
problem, but its output lacks realism, which is not tolerated by users. So the balance between realism and lightweight
is a real problem which are eagerly demanded to solve.

In order to solve this problem, a high-efficiency, low-cost and lightweight-classified tree modeling method is proposed.
Here the lightweight-classified means it can produce different lightweight levels of tree models. And to implement this 
goal, this method will reduce model size on the premise of not losing much realism. This method can also be applied furthur
to automatically adapt to the bandwidth and hardware conditions of the client side.

For implementing the lightweight method, we first improve the traditional PyrLK optical flow method to support affine transformation
and backward feature tracking, which can furthur be applied to do feature matching in gpu accelerated 3D reconstruction and 
improve the match ratio. Then we use flooding algorithm in 3D voxel model and least squares method to discover the tree skeleton and
its radius information. According to the lightweight level the applications require, we reduce the model size by merging the
branches vertically and horizontally respectively. At last we propose a modeling quality evaluation method, which will objectively and 
quantizedly evaluate the restore degree of the tree model.
\end{eabstract}

\ekeywords{Image-based Modeling, Tree Modeling, Light-weight Modeling, PyrLK Optical Flow, Skeleton Extraction}

\makecover


% 目录

\tableofcontents

% 符号对照表
%\input{data/denotation}

%%% 以下索引按需要选择
% 插图索引
%\listoffigures
% 表格索引
%\listoftables
% 公式索引
%\listofequations

%%% 正文
\mainmatter

%%% Local Variables:
%%% mode: latex
%%% TeX-master: t
%%% End:



\chapter{引言}
\label{cha:intro}
\section{背景介绍}
\label{sec:background}
在互联网飞速发展的今天,网络应用已经延伸到生活的方方面面。微博、人人网、
在线购物、在线音乐等已经成为当今人们生活的一部分。面向Web的虚拟现实应用
如WebVR、WebGame、WebGIS等也必然将成为虚拟现实发展的重要方向。树木作为自然
界常见的事物,在各种虚拟现实的场景中出现的频率很高。然而树木形态各异,结构
复杂,给3D建模带来了很大的难度。通常单棵树的数据量已经不小,对于构建一个
树木的聚集场景(如森林)就更加庞大,这容易使得场景负荷变大而产生延迟。因此,
树木建模的质量和效率将直接决定面向Web的虚拟现实应用的成败。

目前的树木的3D建模,主要是通过专业的3D建模工具(3DSMAX、Maya等)进行手工建模。
这种建模方法对建模人员的要求较高,并且需要的时间长。而且这种方法通常最终生成
的是面片信息,要表达一棵形态复杂的树木需要大量的顶点信息,导致最终生成的模型
体积较大,对于需要大批树木的场景,负荷就会变得更大。

目前树木的轻量化建模,从最简单的基于分形,广告牌技术的建模到稍微复杂的基于
规则的建模,都存在一个共同的问题,就是在轻量化的同时,很大程度上舍弃了模型
的真实感和树木本身的形态特征。随着当今应用对真实度要求的升高,这类轻量化的
建模方法已经不能完全满足需求。真实感与轻量化之间的权衡也成为了当今应用需要
考虑的一个重要因素。

本课题基于以上的考虑,从基于图片对树木结构进行完整的恢复,到面向应用需要对
真实感与轻量化进行人工控制,到最后模型重建质量的评估,给出了一套完整的解决
方案。

\section{课题的主要工作}
\label{sec:objective}
本课题的主要工作有:

1. 对PyrLK光流法进行改良,并将其运用于三维重建算法中的特征点匹配步骤,使树木
重建的模型更加准确和精细。

2. 提出了基于三维体素泛洪和空间反向投影的三维重建方法,以连续的体素替代传统
方法不连续的点云,使得后续的骨架抽取步骤更加准确和方便。

3. 提出了基于多方向迭代和步长探索的树木骨架抽取方法。该方法区别于传统的3D瘦化
骨架抽取方法,它只适用于具有分形结构的3D骨架,所以更能够得到准确的树木骨架。

4. 提出了基于用户交互对树木模型进行完善和轻量化,让最终的应用来决定其所需的树木
模型,避免了主观的一味轻量化或一味追求真实感而带来的需求矛盾,将模型的成型延迟
至具体的应用。

5. 提出了基于图像的树木重建质量评估方法,对于建模质量和轻量化过程中真实感的下降
程度给出了函数化和量化的评价依据。


%%% Local Variables:
%%% mode: latex
%%% TeX-master: t
%%% End:



\chapter{方法综述与技术简介}
\label{cha:methodsandtechniques}

\section{树木建模方法综述}
\label{sec:treemodelingmethods}

\section{基于多张图像的三维重建方法简介}
\label{sec:mulimg3drecmethods}

\section{基于点云的骨架抽取方法综述}
\label{sec:treesklextraction}


\chapter{基于图像的树木轻量化3D建模方法}
\label{cha:techroute}

\section{技术路线}
\label{sec:techroute}
本文提出了一套完整的基于图像的树木轻量化3D建模的方法。该方法首先以树木图片
序列作为输入,用经过改进的方法对树木进行三维重建,使三维重建得到的模型精确
度和完整性都得以提高。然后再用基于空间方向迭代和步长探索的方法抽取树木的骨
架,最终再基于用户交互对骨架进行改善与轻量化。

该技术路线旨在实现一个对建模设备和条件要求不高,适用于一般应用的方法。在方便
和简单的基础上,尽可能多的加入自动化,并结合少量用户交互,以实现高效、精确的
树木轻量化3D建模。

该方法的主要步骤如下:

\subsection{基于改进的PyrLK光流法的图像特征匹配}
\label{subsec:match}
本文首先将著名的金字塔LK光流法(PyrLK)进行了改进和扩展,使本来以平面平移作为
运动假设的PyrLK光流法扩展为以平面仿射变换为运动假设,这样便能在两帧以空间角度
旋转而拍摄的图片中,捕捉到空间旋转变换的投影,提高了匹配的完整性和正确性。之后
本文再将得到的匹配点进行反向匹配,只有在一定容错区间内的匹配点对才会被接纳,以
提高匹配算法的鲁棒性。

\subsection{三维重建}
\label{subsec:calibration}
特征匹配完成以后,本文使用了美国华盛顿大学西雅图分校Changchang Wu的可视化运动
恢复工具VisualSFM来完成基于多张图片的树木三维重建。VisualSFM实现了SiftGPU(GPU加速)
和多核的捆集调整(Multicore Bundle Adjustment), 使得相机参数的恢复更加
快速和精确。在这个步骤本文用经过改进的PyrLK光流法的匹配结果替换VisualSFM中的
SIFT特征点匹配文件,进一步地改进了相机参数恢复的准确度和可信度。

%\subsection{基于体素泛洪与空间反向投影的三维重建}
%\label{subsec:3drec}

\subsection{基于三维体素泛洪与线性拟合的三维树木骨架抽取}
\label{subsec:treesklextract}
在完成了三维重建之后,将会得到一个比较完整的树木空间点云模型。本文根据该点云的
空间分布,并结合树木自底向上的自然生长规律和分形的逻辑结构特征,在阈值范围内,
进行三维的体素泛洪,同时向多个子方向进行迭代,不断增加步长来扩大邻域范围。在确定
邻域以后将几个点数比例较大的方向作为分支方向,并用线性拟合的方法确定其精确的分支
方向。同时在迭代过程中及时剔除已经形成枝干的点云,来加速泛洪算法的完成。最终获取到的骨架信息
是含有父子关系的节点信息,相比起3DSMAX等手工工具导出的面片模型,这种逻辑结构的
模型大大的减小了其尺寸,但是由于逻辑结构并没有多少丢失,所以极具真实感。并且这种
结构更便于后续的处理和进一步轻量化。

\subsection{基于用户交互的模型改善与轻量化}
\label{subsec:userinteraction}
由前面方法所得到的树木三维骨架虽然已经是含有父子信息的树木逻辑结构,但是由于前面
的步骤都是自动化生成,所得到的结果不可能100\%地保证符合具体应用的需求。并且前面的
骨架信息虽然比起用面片来表示树木模型已经大大的轻量化了,但是针对实际的应用,本文
还可以根据用户的交互来合并分支,从而进一步对模型进行轻量化,以适用于轻量化要求更
高的应用。

\subsection{建模质量评估}
\label{subsec:qualityevaluation}
对于一个通过建模获得的树木模型,如果没有一个客观的量化评价指标,就无法从客观的角度
反馈树木模型的还原度和各个步骤算法的可行性。本文提出了建模还原度的概念,以模型重建
的还原度来量化的表现建模的质量。该还原度分别计算输入图像序列的信息量、三维重建的还原
度和骨架抽取的还原度。并将它们有机地结合起来形成了最后总的建模还原度。

\section{技术路线图}
\label{sec:techrouteimg}
insertplace...


\chapter{算法}
\label{cha:algorithm}

\section{改良的PyrLK光流法}
\label{sec:pyrlk}

\section{基于体素泛洪与空间反向投影的三维重建}
\label{sec:3drec}

\section{基于多方向迭代与步长探索的三维树木骨架抽取}
\label{sec:sklextract}

\section{基于枝干合并的轻量化处理}
\label{sec:branchcombine}

\section{建模质量评估算法}
\label{sec:qualityevaluation}

\chapter{基于三维体素泛洪与线性拟合的三维树木骨架抽取}
\label{sec:sklextract}
在获取了精确的点云模型之后,出于后续轻量化的考虑,需要将模型的存储方式由
密集的点云转化为逻辑的父子结构。用树形的数据结构来表达现实的树结构,这是很
自然的想法,相对于面片结构,树形结构也是一种更为轻量化的存储方式。每个节点表示
树枝的起点,存储着该节点的空间位置,半径和该节点的父子枝信息以及兄弟信息。一个
节点和它的一个子节点形成一个空间线段,若干空间线段组成一条连续的树枝。

本文从树的生长规律入手,从根节点往子节点生长。生长的依据则为当前节点所在邻域
内的空间点云分布,节点邻域大小由步长来控制,步长会探索式地递增,直到达到了增长的阈值,
邻域大小才确定下来。然后从其点云分布拟合出各个分支的方向,从而生长出新的子节点,
并递归地生长下去直到点云的边界。

\section{点云体素化}
前面三维重建步骤得到的结果是一个点云模型,该模型中的点数量庞大,不适于后续的
邻域搜索,因此我们需要对点云进行体素化处理。所谓体素化,就是将离散的点的数据
组织形式转化为连续的体素的组织形式。它主要分为三步:\\

\begin{itemize}
	\item \textbf{求得点云包围盒}: 即找到包围点云模型中所有点的最小的长方体。如图
		\ref{fig:voxel}(a)。
	\item \textbf{包围盒空间分块}: 根据上一步得到长方体,进行空间分块,每个分块
		为一个小的立方体,即体素。如图\ref{fig:voxel}(b)。
	\item \textbf{点云索引}: 对于每一个非空的体素,进行点云的索引。如图\ref{fig:voxel}(c)。
\end{itemize}

\begin{figure}[H]
	\centering
	\subfloat[求得点云包围盒]{\includegraphics[width=0.3\linewidth]{box.png}}\hfill
	\subfloat[包围盒空间分块]{\includegraphics[width=0.3\linewidth]{voxel.png}}\hfill
	\subfloat[点云索引]{\includegraphics[width=0.3\linewidth]{index.png}}
	\caption{点云体素化}
	\label{fig:voxel}
\end{figure}

在将点云模型转化为体素模型以后,对于点云的邻域搜索便转化为了对于空间临近体素的
搜索,体素的位置就反映了点集的位置,因此不用每次搜索都遍历整个点云,而是只用将
步长范围体素中的点集遍历即可。由于体素是我们处理的基本单位,所以体素的大小也
直接决定了体素模型的精度,因此,在确保非空体素的空间连续性和效率允许的基础上,
本文建议让体素尽可能的小,以保证模型的精度。将点云模型转换为体素模型的伪代码
在算法\ref{alg:voxel}中给出。

\begin{algorithm}[H]
	\caption{点云模型体素化}
	\label{alg:voxel}
	\begin{algorithmic}[1] 
	\Require 点云模型$M$
	\Require 体素维度$d$
	\Ensure 三维体素数组$\mathbb{V}[1..d,1..d,1..d]$
	\State 初始化点云边界值$X_{max}=Y_{max}=Z_{max}=MIN\uline\quad FLOAT,X_{min}=Y_{min}=Z_{min}=MAX\uline\quad FLOAT$
	\ForAll{空间点$P(P_x,P_y,P_z) \in M$}
		\State $CheckBoundary(P)$
	\EndFor
\ForAll{空间点$P(P_x,P_y,P_z) \in M$}
\State $V_x = \frac{P_x-X_{min}}{X_{max}-X_{min}}\cdot d$
\State $V_y = \frac{P_y-Y_{min}}{Y_{max}-Y_{min}}\cdot d$
\State $V_z = \frac{P_z-Z_{min}}{Z_{max}-Z_{min}}\cdot d$
\State $\mathbb{V}[V_x, V_y, V_z] = \mathbb{V}[V_x, V_y, V_z] \bigcup \{P\} $
	\EndFor
\end{algorithmic}
\end{algorithm}

\section{三维体素泛洪确定邻域范围}
在确定了三维体素模型以后,便需要从根到叶,自底向上地对树的骨架结构进行生长。
生长的依据是已经得到的体素模型,将体素模型中点的分布作用于骨架的分支,便可以
张成骨架模型。

具体方法是将根节点置为当前节点,对其进行三维泛洪,首先对其相邻的26个体素进行泛洪,若
体素不为空,则将其加入邻域范围,若为空,则停止向该方向进行迭代。同时将加入邻域
范围的体素置为无效,表示其已经参与了泛洪,不再参与骨架的重建,这样不仅可以对算法
的结束有一个很好的约束条件,同时也可以减少重复处理的次数,加快算法的完成。然后进行下一次迭代,
对新加入的体素进行26方向的泛洪,并把有效的体素加入到邻域范围。接着比较两次迭代
体素增加的比例,如果低于设置的阈值,则停止迭代,当前的邻域范围即为三维泛洪得到
的当前节点的邻域范围。

图\ref{fig:3dfld}展示了三维体素泛洪确定邻域的步骤,三张图都延空间z轴正向投影到2D平面。
\ref{fig:3dfld}(a)为其初始状态,
即邻域范围为当前体素。其中橙色的区域表示邻域范围,蓝色的区域表示未探索区域,灰色区域
表示空的体素,而绿色区域表示已经在之前的枝干邻域。\ref{fig:3dfld}(b)表示体素泛洪经过
一次迭代以后的状态,因为体素泛洪只会对与当前邻域范围相邻的未探索区域(蓝色方块)进行扩展,
所以\ref{fig:3dfld}(a)只会向黄色箭头指向的体素进行扩展,从而得到\ref{fig:3dfld}(b)。在得到
新的邻域后,首先会计算所新增的点的数量与之前的数量的比值有没有低于阈值,如果低于阈值,则停止
邻域的扩张。最后将得到\ref{fig:3dfld}(c)中的邻域范围。

\begin{figure}[H]
	\centering
	\subfloat[初始状态(邻域范围为1个体素)]{\includegraphics[height=3cm]{fld1.jpg}}\hspace{4em}
	\subfloat[第一次泛洪迭代(邻域范围为6个体素)]{\includegraphics[height=3cm]{fld2.jpg}}\hspace{4em}
	\subfloat[第二次泛洪迭代(邻域范围为11个体素)]{\includegraphics[height=3cm]{fld3.jpg}}
	\caption{单个体素泛洪示意图}
	\label{fig:3dfld}
\end{figure}

由于树木有多个节点,所以确定节点间的泛洪顺序十分重要。因为泛洪算法对泛洪过的区域不再进行泛洪,因此
需要将各个节点之间的相互影响降到最低。本文通过广度优先的方法,按层级对体素进行泛洪,这样就避免了子
节点的泛洪影响到叔父节点的泛洪。同时,对于同一层级的体素,将其视为多个种子点,并采取并发的泛洪,
也就是同时对它们进行泛洪,这样既提高了泛洪的效率,也使得同层级间的体素之间的影响降到最小。

图\ref{fig:flood}给出了多种子点并发泛洪确定邻域的示意图。其中蓝色方块为未泛洪体素,着红色的方块表示
已泛洪体素,橙色的小球为当前的种子点。由图可见,当前的种子点为树木的相同层级上的节点,即广度优先的
并发泛洪。

\begin{figure}[H]
	\centering
	\subfloat[步骤1]{\includegraphics[width=0.4\linewidth]{seed1.png}}\hfill
	\subfloat[步骤2]{\includegraphics[width=0.4\linewidth]{seed2.png}}\hfill
	\subfloat[步骤3]{\includegraphics[width=0.4\linewidth]{seed3.png}}\hfill
	\subfloat[步骤4]{\includegraphics[width=0.4\linewidth]{seed4.png}}\hfill
	\caption{多种子点并发泛洪示意图}
	\label{fig:flood}
\end{figure}

多种子点并发体素泛洪确定邻域范围算法的伪代码在算法\ref{alg:3dfld}中给出。
\begin{algorithm}[H] 
	\caption{多种子点并发泛洪邻域探索算法}
	\label{alg:3dfld}
	\begin{algorithmic}[1]
		\Require 当前层级体素集$\mathbb{C}$,三维体素数组$\mathbb{V}[1..d,1..d,1..d]$,
		泛洪方向数组$\mathbb{D}[1..26]$,邻域范围增长比例阈值$\lambda$,最大迭代次数$num$
		\Ensure	邻域范围内体素集合$\mathbb{S}$
		\State 当前层级体素数目$n$ = $\mathbb{C}$.Count
		\State 初始化前一次泛洪体素数目数组$\mathbb{P}[1,..,n]$
		\For{i = 1 \textbf{to} $num$}
		\State 体素索引$j = 0$
		\ForAll{体素$voxel \in \mathbb{C}$}
		\State $j++$
		\State 临时体素数组$\mathbb{T}$
		\ForAll{泛洪方向$Direction \in \mathbb{D}$}
			\State $NewIndex = voxel.Index + Direction$
			\State $NewVoxel = \mathbb{V}[NewIndex.x,NewIndex.y,NewIndex.z]$
			\If{$NewVoxel$非空$\bigcap NewVoxel$有效}
				\State $\mathbb{T}.AddVoxel(NewVoxel)$
			\EndIf
		\EndFor
		\State 体素增长比例$\mu=\frac{\mathbb{T}.Count}{\mathbb{P}[j]}$
		\If{$\mu > \lambda$}
			\ForAll{$voxel \in \mathbb{T}$}
				\State S.AddVoxel(voxel)
			\EndFor
		\EndIf
		\EndFor
		\EndFor
		\State \Return $\mathbb{S}$
	\end{algorithmic}
\end{algorithm}

\section{最小二乘法线性拟合确定分支方向}
\label{subsec:leastsquares}
当得到邻域范围以后,便得到了邻域内体素在基于当前节点26个方向上的密度分布,而
每个体素内又包含着若干的点,因此等于是得到了在当前节点邻域内的点云分布情况。
接下来的工作就是怎样从各个方向的点云的分布情况抽取出核心的骨架。本文应用线性
拟合的方法来从密集的点中抽取出一条直线,作为该部分的骨架方向。

该方法首先要剔除掉那些点云密度很小的方向,以免每个节点都朝各个方向长出一些
细碎的枝条。因为这些细碎的枝条就算在此步中不剔除,到后续的轻量化的时候也不容许
它们的存在。

然后对于剩下的若干方向$d_1,d_2...d_k$,每个方向都对应着树木的一个骨架。在处理
某个方向$d_i$时,将其包含的体素中的所有点抽取出来,得到一个密集的点集$S_i$。
然后采用待定方程的办法,设直线方程为:
\begin{equation}
	\mathbf{x} = \mathbf{x_0} + \mathbf{d}t,\quad(t \in [0,\infty))
\end{equation}

其中$\mathbf{x_0}$是当前节点的坐标,$\mathbf{d}$是待拟合的直线方向。我们假设
点集$S_i$中的点$P_1,P_2,...P_m$都在直线上,则可以得到以下方程组:\\

\begin{equation} \label{eq:line}
	\left\{ 
		\begin{array}{lll}
			a_{11}d_x+a_{12}d_y+a_{13}d_z & = & b1\\
			a_{21}d_x+a_{22}d_y+a_{23}d_z & = & b2\\
			... & & \\
			a_{n1}d_x+a_{n2}d_y+a_{n3}d_z & = & bn
		\end{array}
	\right.
\end{equation}

其中具体数值未给出,注意这里的$n=3m$,因为每个点$P$可以提供三个方向的方程式。
在这个方程组中,令\\
\begin{displaymath}
	\mathbf{U}=
\left(
\begin{array}{ccc}
	a_{11} & a_{12} & a_{13}\\
	a_{21} & a_{22} & a_{23}\\
	... & ... & ...\\
	a_{n1} & a_{n2} & a_{n3}\\
\end{array}
\right)
,\quad
\mathbf{d}=
\left(
\begin{array}{c}
	d_x\\
	d_y\\
	d_z
\end{array}
\right)
,\quad
\mathbf{b}=
\left(
\begin{array}{c}
	b_1\\
	b_2\\
	...\\
	b_n
\end{array}
\right)
\end{displaymath}


在实践中,由于筛选方向上的点数较多且发散分布,由线性代数的理论知,$\mathbf{U}$是过约束的,
即$n>r$,其中$r$是矩阵$\mathbf{U}$的秩。这种情况下没有标准的解,只能找到使误差最小的向量$\mathbf{d}$,
误差定义为:\\
\begin{equation}
	E\xlongequal{def} \sum_{i=1}^n(\mathbf{d}t_i - \mathbf{x_i} + \mathbf{x_0})^2=|\mathbf{Ud}-\mathbf{b}|^2
\end{equation}

由于$E$正比于方程的均方误差,因此只要E达到最小值,那么点集相对于该直线的波动就最
小。换句话说,也就是该直线最好的模拟了该点集所表示的骨架。由线性代数的方法很容易
可以解得$\mathbf{d}=\mathbf{[(U^TU)^{-1}U^T]b}$。图\ref{fig:fitting}展示了由当前
节点(蓝色节点)分别向两个点云集合拟合出的两条直线(红色线段),这两条直线将被作为两个
分支的方向。从图中可以看出线性拟合的方法可以很好的估计出树木分枝的方向,从而准确的
恢复出树木的父子结构。

\begin{figure}[H]
	\centering
	\includegraphics[height=5cm]{branch.png}
	\caption{线性拟合计算分支方向}
	\label{fig:fitting}
\end{figure}

算法\ref{alg:sklextract}给出了得到邻域信息后进行骨架方向抽取的伪代码,其中\textit{Least Squares Processing}表示运用最小二乘法进行
线性拟合。

\begin{algorithm}[H]
	\caption{基于邻域的骨架方向计算}
	\label{alg:sklextract}
	\begin{algorithmic}[1] 
		\Require 当前节点体素$V$
		\Require 骨架方向数组$\mathbb{D}[1..n]$
		\Ensure 当前节点子节点集合$\mathbb{S}$
		\ForAll{骨架方向$d\in D$}
			\State $NewChild\gets Least Squares Processing$
			\State $\mathbb{S}.AddChild(NewChild)$
		\EndFor
		\State \Return $\mathbb{S}$
	\end{algorithmic}
\end{algorithm}

\section{拟合子枝长度和半径}
树木骨架的长度和半径对树木模型的真实感有着十分显著的贡献,所以尽可能准确的获得子枝的长度和
半径信息能够有助于重建出极具真实感的树木模型。对于树木枝干半径的获取方法有许多,
主要分为根据规则生成半径和从树木点云结构中获取半径两种方式。

对于基于规则来生成半径,最简单的方法是对树木半径进行线性地递减,即$r=cR$,其中
$r$为子枝半径,$R$是父枝半径,$c$为一个线性倍数,这个倍数可以固定,也可以进行
随机的扰动从而增进多样性。Leonardo da Vinci在经过大量观察后总结出了一种更符合自然规律
的树木父子枝直径的关系公式:$D^2=\sum_{i=1}^n{d_i^2}$,其中$D$为父枝直径,$d_i$为第
$i$个子枝的直径,$n$为子枝的数量。这个公式被广泛地用于树木枝干的半径模拟。

区别于基于规则的半径生成方法,本文为了进一步提升真实感,选择在进行子枝方向抽取的同时,
同样进行半径抽取的方法。注意,用该方法的前提是点云分布须均匀化,然而基于图像进行三维重建
得到的树木点云会呈现表皮化的现象,这是由于图片上的点都是树木的表皮点,所以在得到
三维点云后,是需要进行一些修复工作的,本文用随机点填充的方法对该点云模型进行了实心化
的修复。当点云分布满足均匀化时,在对某个骨架进行拟合之后,对于拟合出来的直线,根据点
到直线的距离,可以同样根据最小二乘法拟合出骨架的半径。
该算法的伪代码在算法\ref{alg:radius}中给出。\\

\begin{algorithm}[H]
	\caption{半径拟合算法}
	\label{alg:radius}
	\begin{algorithmic}[1] 
		\Require 拟合出的当前子枝所在直线$L$
		\Require 当前子枝的点集$\mathbb{S}$
		\Ensure 当前子枝半径$R$
		\State 初始化点到直线距离数组$\mathbb{D}[1,..,n]$
		\For{$i$ = 1 \textbf{to} $n$}
		\State 点到直线距离$\mathbb{D}[i]=CalculateDistance(P, L)$
		\EndFor
		\State 拟合出使得下面式子达到最小值的半径$R$:\[ \sum_{i=1}^n (2\times \mathbb{D}[i] - R)^2  \]
		\State \Return $R$
	\end{algorithmic}
\end{algorithm}

图\ref{fig:radius}给出了三种半径求解方法的效果对比。\ref{fig:radius}(a)给出了线性衰减方法
的结果,该方法中子枝半径以父枝半径的线性倍衰减。\ref{fig:radius}(b)给出了前文提到的Leonardo
 da Vinci规则所生成的半径情况。\ref{fig:radius}(c)则采用本文中基于线性拟合的方法。
 从三者的效果中可以看出,线性衰减容易出现部分枝条生长不自然的现象,究其
 原因,还是因为一个单一的绝对的线性系数无法适用于所有的枝条,它对于某些枝条会偏大,对于另外一些
 枝条会偏小。 Leonardo规则虽然给出的是一种父子枝之间的相对关系,从一定程度上解决了线性系数单一
 绝对而导致的问题,但是它生成的树木枝干会出现过于均与化,而没有捕捉到现实中树木各个局部的特征。
 本文的方法则由于其基于对所有点的实际恢复坐标进行统计,而更加注重树木的实际局部特征情况,
 其效果也是三者之中最好的。
 \begin{figure}[H]
	\centering
	\subfloat[样本图像]{\includegraphics[height=6cm]{rsample.jpg}}\hspace{4em}
	\subfloat[线性衰减(线性衰减系数为0.6)]{\includegraphics[height=6cm]{rlinear.png}}\hspace{4em}
	\subfloat[Leonardo规则生成($D^2=\sum_{i=1}^n{d_i^2}$)]{\includegraphics[height=6cm]{rsquare.png}}\hspace{4em}
	\subfloat[基于拟合的半径抽取]{\includegraphics[height=6cm]{raffine.png}}\hspace{4em}
	\caption{三种计算半径方法效果对比}
	\label{fig:radius}
\end{figure}

对于骨架长度的估计,基于规则的生成则并不那么具有实践性,因为树木枝干的长度往往并不像半径那样
随着父子关系而递减。相反地,它地规则往往要复杂许多,而且并没有统一的规则。基于此考虑,本文并
没有对基于规则的长度估计进行实践,而是直接用与半径类似的方法,根据已拟合出的直线,试图从统计
的角度对其枝条长度做出合理的估计。一个直观而可行的方法,是将当前节点与所有该方向邻域的点的连线向量
投影到拟合出的直线方向向量上,然后根据拟合的方法得到高精确度的长度信息。
子枝长度拟合的伪代码,列在了算法\ref{alg:length}中。

\begin{algorithm}[H]
	\caption{子枝长度拟合算法}
	\label{alg:length}
	\begin{algorithmic}[1] 
		\Require 拟合出的当前子枝方向向量$L$,父节点$N$
		\Require 当前子枝的点集$\mathbb{S}$
		\Ensure 当前子枝长度$l$
		\State 初始化父节点到邻域点的向量投影长度数组$\mathbb{D}[1,..,n]$
		\For{$i$ = 1 \textbf{to} $n$}
		\State 父节点到$\mathbb{S}[i]$的向量$Dir=\mathbb{S}[i]$.Position-$N$.Position
		\State $Dir$在$L$上的投影距离$\mathbb{D}[i]$ = CalculateProjectionLength$(Dir, L)$
		\EndFor
		\State 拟合出使得下面式子达到最小值的长度$l$:\[ \sum_{i=1}^n (2\times \mathbb{D}[i] - l)^2  \]
		\State \Return $l$
	\end{algorithmic}
\end{algorithm}

\section{本章小节}

本章首先将三维重建得到的点云数据进行体素化,通过对体素求解包围盒、空间分块、点云索引的方法,将离散的点云数据
转化为连续的体素数据。

接着本章提出了基于新的树木骨架抽取方法,主要包含:\\
\begin{itemize}
	\item \textbf{多种子点的并发三维体素泛洪}: 该方法通过广度优先的方法,对树木节点进行遍历。同时对于同一层级
		上的节点,将它们作为种子点,并并发的进行三维体素的泛洪,以将节点间的相互影响降至最低。
	\item \textbf{最小二乘法拟合骨架信息}: 该方法通过最小二乘的方法,对空间的局部点云进行方向、半径和长度的
		拟合,从而以统计的方法从大量的空间点数据中抽取出有用的骨架信息。
\end{itemize}

通过本章的方法抽取出的骨架具有很高的准确性和真实感,因为它是通过统计的方法在实际的点云中进行信息的获取,相比
起其他的基于规则的信息抽取方法,该方法有明显的优势。

%

\chapter{基于枝干合并的轻量化处理}
\label{cha:branchcombine}
用基于多方向迭代与步长探索得到的三维树木骨架通常是很细致和准确的,尽管它相对于
用3DSMAX等建模工具手工建模得到的面片模型已经大大的轻量化了。但是如果应用是用于
大规模的树木建模,那么我们有必要根据应用需求进一步进行轻量化处理。

\section{L-System的尝试}
\label{subsec:lsystem}

\subsection{L-System简介}
L-System是一种并行的重写系统和正规语法,
它的结构可以用可以定义为一个3元组:\\
\[\mathbf{M} = (V, \omega, P)\]
其中:\\
\begin{itemize}
	\item $\mathbf{V}$(字母表) 表示可以被替代的字符的集合。
	\item $\mathbf{\omega}$(初始串) 表示L-System的初始状态。
	\item $\mathbf{P}$(规则集合) 表示一系列的衍生规则。
\end{itemize}
L-System可以根据这三个组成部分的不同而递归地产生形态各异的字符串。
由于L-System具有递归生长的特性,因此我们可以用L-System规则来表达一个具有自相似形态
或者分形结构的物体,比如本文所研究的对象\raisebox{0.5mm}{------}树木。

\subsection{树木模型的参数化L-System规则抽取}
球面海龟几何的提出,用参数化的L-System规则描述了树木的结构信息。在球面海龟几何中,
节点的空间几何信息用4个量(长度$l$、半径$r$、父子枝夹角$\theta$和水平转角$\phi$)
和4个扩展符号(+、-、\&、$\wedge$)来表示:
\begin{itemize}
	\item $+(l)$	表示以当前位置为起点,在当前方向上前进$l$单位个长度
	\item $!(r)$	表示设置当前节点半径为$r$
	\item $\&(\theta)$	表示设置父子枝夹角为$\theta$
	\item $\wedge(\phi)$	设置水平偏角为$\phi$
\end{itemize}
在球面海龟几何中,将每个骨架节点生成一条参数化的L-System规则,形如:\\
\begin{equation} \label{eq:turtle}
N(l,r) \rightarrow \&(\theta_0)\wedge(\phi_0)!(r) + (l)S_0(l*a_0,r*b_0)...\&(\theta_n)\wedge(\phi_n)!(r) + (l)S_n(l*a_n,r*b_n)
\end{equation}

其中N表示当前枝条,$S_0~S_n$表示当前枝条的n个子枝条,$a_i和b_i$分别表示第i个子枝条
与当前枝条的长度比和半径比,$\theta_i和\phi_i$分别表示第i个子枝条与当前枝条的空间
夹角和水平偏角。

\subsection{使用L-System进行树木轻量化建模遇到的问题}
在用参数化L-System进行树木轻量化建模时,在进行规则归纳时,有个难以克服的问题。考虑
将规则\ref{eq:turtle}中的$a_0$换成$a_0'$,则规则变成一个完全不同的规则。这意味着对于
两个分支规则,这两个规则中的子枝的长度,半径,转交,偏角等必须完全相等才能归纳为同
一个规则。而对于自然界中形态结构复杂的树木,每个分支规则几乎不可能完全等同于另一个
规则。

对上面的问题有一种解决方法就是将参数区间化,将属于同一区间的参数的值视为相同。比如
我们可以将父子枝间的转角分为18个区间,每个区间的大小为10度。但是经过分析就可以察觉,
这并没有从根本上解决这个问题。假设我们将这4个变量都各自划分为10个区间,那么规则总数
最多可以有10000个,而且在这种情况下,两个规律相同的几率也是非常小的。如果我们将分区
数量减少,则又有可能将本来差异比较大的规则归纳为一个规则,不符合真实感的要求。

所以,经过分析,这种用参数化L-System进行树木轻量化建模的方法并不适用于从骨架中去抽取
规则,而是适用于反向地用其描述的规则去产生一棵树,如台湾学者戴文凯就对单棵树的L-System
规则进行随机扰动而轻量化的建模出了整片森林。

\section{基于枝干合并的树木分级轻量化}
\label{subsec:branchmerge}
用L-System的方法抽取规则所产生的问题,从本质上看,是由于自然界中的树木形态太复杂和多变。
与其从一个本就不规则生长的事物中去抽取规则,还不如直接地在其逻辑结构上进行一系列的轻量
化操作。本文提出了分级化地对已抽取的树木骨架中对视觉影响不大的部分进行合并的方法,从而在尽可能
保证模型的视觉效果的基础上,进一步地减小树木模型的体积,使得其能更广泛地应用到WebVR、WebGame
等各个领域。

树枝的结构其实只由核心的一些枝干组成,其他的枝干只是对其结构进行微调。所以在要求进一步轻量化
的前提下,本文提出了分别从纵向和横向对树枝进行合并的方法,以去掉一些只是起到微调作用的枝干。
这种方法在尽可能保证真实感不过多丢失的前提下对树枝进行简化操作,以适应更广泛的Web应用领域。

\subsection{基于夹角的枝干纵向合并算法}

纵向合并表示从父到子,从根到页进行纵向递归式的合并。若当前节点与其父节点和子节点的夹角小于所设定
的阈值,那么则将该节点去掉,并将其子节点连接到其父节点。注意,若该节点的子节点数目不只一个,
那么我们不对它进行合并操作,因为将该节点的所有子节点加到该节点的父节点上去有违真实感。
图\ref{fig:vert}展示了树枝纵向合并过程。图\ref{fig:vert}(a)为输入的树枝骨架,并且当前节点
为\textbf{B},其父节点为\textbf{A},且只有唯一的子节点\textbf{C}。设合并角度阈值为$\alpha$,
假设\textbf{AB,BC}之间的夹角b小于合并阈值$\alpha$,那么将\textbf{B}剔除,并将\textbf{C}作为
\textbf{A}的子节点。同理,在图\ref{fig:vert}(b)中,若夹角c小于阈值$\alpha$,那么也将\textbf{AC}
和\textbf{CD}合并。在图\ref{fig:vert}(c)中,由于节点\textbf{D}有两个孩子,所以不对其进行合并
操作。

基于夹角的枝干纵向合并算法的伪代码在算法\ref{alg:vertical}中给出。

\begin{figure}[H]
	\centering
	\subfloat[合并AB、BC]{
\includegraphics[height=5cm]{vert1.jpg}}
\hspace{4em}
	\subfloat[合并AC、CD]{
\includegraphics[height=5cm]{vert2.jpg}}
\hspace{4em}
	\subfloat[合并完成]{
\includegraphics[height=5cm]{vert3.jpg}}
	\caption{树枝简化过程}
	\label{fig:vert}
\end{figure}

\begin{algorithm}[H]
	\caption{基于夹角的枝干纵向合并}
	\label{alg:vertical}
	\begin{algorithmic}[1] 
		%\Comment {根据纵向合并角度参数,以当前节点为发起点递归式地纵向合并枝干}
		\Require 纵向合并角度$\alpha$
		\Require 当前节点$N$
		\Ensure None
		\ForAll{节点$N'\in N.Children$}
		\While{$N'.ChildCount = 1$}
		\State $\vec{u} \gets N'.Position-N.Position$
		\State $\vec{v} \gets N''.Position-N'.Position$
		\State $\gamma \gets \cos^{-1}({\frac{\vec{u} \cdot \vec{v}}{|\vec{u}|\cdot|\vec{v}|}})$
		\If{$\gamma<\alpha$}
		\State $N.child \gets N.AddChild(N'')$
		\State $N.child \gets N.DeleteChild(N') $
		\EndIf
		\State $N' \gets N'.FirstChild$
		\EndWhile
		\EndFor
		\If{$N.ChildCount > 1$}
		\ForAll{节点$N'\in N.Children$}
		\State 以$N'$为当前节点递归调用该函数
		\EndFor
		\EndIf
	\end{algorithmic}
\end{algorithm}

\subsection{基于端点距离的枝干横向合并算法}

横向合并指的是对非常靠近的叶子节点进行合并。之所以只对叶子节点进行合并,是
因为非叶子节点下面都有若干棵子树,若对它们进行合并,必须对它们下面的子树也进行合并。而合并子树
显然就使得真实感下降很大,因为这不只是局部微调,而是若干子树的变动。对于横向合并,不再是使用角度
来衡量两个子枝的靠近程度,而是使用子节点间的欧式距离来表示,因为合并两个角度相差小但是长度相差大的
子枝也会导致真实感的大幅下降。横向合并的伪代码在算法\ref{alg:hori}中
给出。

\begin{algorithm}[H]
	\caption{基于端点距离的枝干横向合并}
	\label{alg:hori}
	\begin{algorithmic}[1] 
	\Require 初始化横向合并距离阈值$\mu$
	\Require 设定当前节点$N$
	\Ensure None
	\ForAll{节点对$P\in N.Children$}
		\State $N_1 \gets P.FirstNode$
		\State $N_2 \gets P.SecondNode$
		\If{$N_1.ChildCount = 0 \wedge N_2.ChildCount = 0$}
		\State $\vec{\mathbf{u}} \gets N_1.Position$
		\State $\vec{\mathbf{v}} \gets N_2.Position$
		\State $\gamma \gets |\vec{\mathbf{u}} - \vec{\mathbf{u}}|$
			\If{$\gamma<\mu$}
				\State $New\ Node\ N'$
				\State $N'.Position \gets (N_1.Position+N_2.Position)/2$
				\State $N'.Radius \gets max(N_1.Radius,N_2.Radius)$
				\State $N.child \gets N.DeleteChild(N_1)$
				\State $N.child \gets N.DeleteChild(N_2)$
				\State $N.child \gets N.AddChild(N')$
				\State 退出循环并以当前节点N重新调用该函数
			\EndIf
		\EndIf
	\EndFor
	\ForAll{节点$P\in N.Children$}
		\State 以P为当前节点递归调用该函数
	\EndFor
\end{algorithmic}
\end{algorithm}

\subsection{综合使用横纵向合并}

纵向合并和横向合并单独使用时都具有很大的局限性,因为纵向合并只能对具有单个孩子并且没有
兄弟的节点进行纵向递归地调用,而横向合并又只能对叶子节点进行兄弟级别的合并。但是将两种
合并方法联合使用,将可以从整体上对树木进行微调操作,图\ref{fig:combine}对这一想法进行了
演示。\ref{fig:combine}(a)中经过纵向的\textbf{AB,BC}合并得到\ref{fig:combine}(b)。
\ref{fig:combine}(b)中由于\textbf{C}有两个子节点,无法进行纵向合并,所以考虑进行横向
合并\textbf{CD,CE},并得到\ref{fig:combine}(c)。最后进行一次纵向合并得到\ref{fig:combine}(d)。

\begin{figure}[H]
	\centering
	\subfloat[纵向合并AB,BC]{
	\includegraphics[height=6cm]{comb1.jpg}}
	\hspace{6em}
	\subfloat[横向合并CD,CE]{
	\includegraphics[height=6cm]{comb2.jpg}}
	\hspace{6em}
	\subfloat[纵向合并AC,CE]{
	\includegraphics[height=6cm]{comb3.jpg}}
	\hspace{6em}
	\subfloat[合并完成]{
	\includegraphics[height=6cm]{comb4.jpg}}
	\caption{联合使用纵向和横向合并}
	\label{fig:combine}
\end{figure}

\section{本章小节}
在得到骨架结构后,为了迎合轻量化的应用,本文对其进行轻量化操作。本章首先对传统的轻量化方法L-System进行了尝试,运用参数化
的L-System规则进行规则抽取,但是由于现实中树木的复杂性与不规则性,抽取出来的规则太多,以至于违背了轻量化的原则。于是本文
提出了基于枝干合并的轻量化方法,分别从纵向和横向对树木枝干进行合并,并对这两种方法进行有机的组合使用,以达到轻量化的目标。

% chap07.tex

\chapter{建模还原度度量}
\label{sec:qualityevaluation}
对于一个通过建模获得的树木模型,如果没有一个客观的量化评价指标,就无法从客观的角度
反馈树木模型的还原度和各个步骤算法的可行性。对于本文的基于图像序列的树木建模方法,
建模的输入是在自然环境下拍摄的树木图片序列,输出是三维的骨架模型。因此,判断三维模型
和投影照片的相似程度是评价建模质量的核心。然而,大多数基于图像的树木建模论文\cite{quanlong,
tanping,lichuan,tanping2,liu}只给出了输入图片和建模结果在少量角度的渲染效果,试图让
观察者从肉眼观察其相似度。但是这种方法是主观的,因观察者的不同可能会有不同的评价结果,
这显然不是一个好的评价方法。

为了客观、量化地评价基于图像序列的建模质量,本文提出了一套完整的评价方法。然而,仅仅凭借照片
无法完全表达出其所在环境的信息,比如环境光照,因为遮挡而产生的阴影信息等,因此本文的评价
方法将不针对模型的纹理和颜色信息,仅仅对模型的几何信息和照片中的几何信息的匹配程度进行量化
分析。

设树木模型$M$由$n$张从不同角度拍摄的同一棵树的图片序列$I_1\sim I_n$,经过基于图像的三维重建,
骨架抽取的方法进行建模所获得。那么模型$M$的建模还原度$\mathbb{Q}$定义如下:\\
\begin{definition}
	\[\mathbb{Q}=\mathbf{I}\cdot\mathbf{R_{3d}}\cdot\mathbf{R_s}\]
\end{definition}

建模还原度$\mathbb{Q}$的取值范围为$[0,1]$,0表示没有还原出任何树木几何信息,1表示准确还原出
整棵树木的几何信息。

本文将建模还原度$\mathbb{Q}$考虑由3个部分组成,为此也引入了三个新的概念:图片序列信息量$\mathbf{I}$,
三维重建还原度$\mathbf{R_{3d}}$,以及骨架抽取还原度$\mathbf{R_s}$。这三个分量的取值范围都为$[0,1]$,它们
的乘积即为总的建模还原度$\mathbb{Q}$。后续小节会详述这三个分量。

\section{图像序列信息量}
当实地对树木进行多角度拍摄时,拍摄者将基于不同的水平角度对树木进行全方位的拍摄,以便将整棵树的
信息尽可能多的携带进图像序列中。然而,从客观上来看,怎么样的图片序列才更加完整的表达了整棵树的
几何特征?为了从客观和量化的角度给出树木图像序列所携带的树木信息的多少,本文引入了图像序列信息量的概念。

那么,到底怎样的图片序列携带的信息量更大呢?从拍摄过程分析,如果想要得到一棵完整的树木信息,那么
需要绕着一棵树一圈进行密集地拍摄。这里的一圈,用数学化的表示,就是$360^\circ$,如果只是绕半圈进行
拍摄,那么所得到的图片序列表达的树木信息必定是不完整的,所以角度对信息量有着很大的贡献。另一方面,
如果每隔$60^\circ$进行一次拍摄,和每隔$30^\circ$进行一次拍摄,在它们都绕圈拍摄的前提下,后者的图片序列所
含信息量必然更大。再进一步思考,如果我隔$360^\circ,180^\circ,90^\circ,..., 1^\circ$进行拍摄呢?那么后一次拍摄所得的图
片序列相比前一次的图片序列信息量的增长是相同的吗?答案是否定的,因为当图片很少时,三维重建的结果也不好,
这时增加图片数量是能够很大程度上改观三维重建的重量的,因此此时的信息量增长速度快。
但是在拍摄已经比较密集的情况下,后一次拍摄所增加的信息量必定只是一些细节的信息,所以,信息量增长的速率应该变小。
并且一个信息量大的图片序列应该满足以下三个要求:\\
\begin{itemize}
	\item \textbf{图片数量多}: 图片数量多也就意味着拍摄角度多,因为一张图片代表着一个角度。
	\item \textbf{角度跨度大}: 跨度大指需要对树木进行全方位的拍摄。
	\item \textbf{角度分布均匀}: 若图片只是密集的集中在一个角度区间,就算图片再多,也无法完整地表达整
								  棵树的信息,所以若在角度多和跨度大的情况下还满足分布均匀,那么就能很完整
								  地携带树木的信息。
\end{itemize}

由于从平面的2D图像很难得到其空间角度拍摄情况,因此在这里我们简化其定义,将关注点放在图片数量上来,对于
图片跨度和角度的均匀分布,我们默认拍摄者在拍摄过程中采用均匀的角度偏差来进行$360^\circ$的拍摄。

根据以上的分析,本文给出了图像序列信息量的数学定义如下:\\
\begin{definition}
	\[ \mathbf{I}=1-\left[\frac{a}{b}\right]^n \]
\end{definition}

其中,图片序列信息量$\mathbf{I}$的取值范围为$[0,1]$。当$\mathbf{I}=0$时表示图片序列并不包含树木信息,
当$\mathbf{I}=1$时表示图片序列能完全表达空间树木的几何信息。$a,b$都是正数且$a<b$,具体数值需要对不同树木
进行实验之后才能得到。尽管$a$和$b$因树木特点不同而不同,但是它始终满足前文提出的信息量增长速度的特点,
即先快后慢。

\section{三维重建还原度}
对于一个给定的图片序列,所用三维重建方法所得到的点云模型的与实际的树木在几何形状上的相似度如何,由三维重建
还原度$\mathbf{R_{3d}}$来定义。注意,实际树木的几何信息被记录在输入的图像序列中,所以想要计算点云模型和实际树木
的相似度,就需要对点云模型和图片序列进行比较。然而对于三维的点云信息和二维的图片信息,无法进行直接地比较。一个
比较直观的想法,是对三维的点云进行投影,投影的角度由三维重建过程中的照相机几何标定步骤给出。

由于不考虑模型纹理和颜色信息,在空间点被投影到平面以后,只关注其是否在对应角度图片的树木轮廓内。所以对输入的树木
图片序列,需要先获得其轮廓图,并将其转化为二值图像。树木上的点值为1,而树木外的点值为0。对于每一个点云模型中的点,
按对应角度投影,获得其在对应图片上的坐标值,并且在其二值图像上确定其值,若为1,则表明匹配成功,否则匹配失败。最后
统计出匹配成功的总的比例,作为三维重建的还原度。

根据以上分析,本文给出三维重建还原度的数学定义式:\\
\begin{definition}
	\[ \mathbf{R_{3d}}=\frac{1}{n}\sum_{i=1}^n \frac{P_i}{P_i+O_i}\]
\end{definition}

上式中的$n$表示图像的数量,$P_i$表示点云模型投影到第$i$张图片上在树木轮廓中的点的数量,$O_i$表示点云模型投影到第
$i$张图像上在树木轮廓外的点的数量,因此$P_i+O_i$自然就表示点云模型中点的总数量。$\frac{P_i}{P_i+O_i}$表示点云投影到
第i张图片上的击中率。最后对每张图像的击中率求平均,作为总的三维重建的还原度。其值区间为$[0,1]$。

\section{骨架抽取还原度}
骨架抽取是基于三维点云模型进行的,因此计算骨架抽取的还原度的输入是重建出的点云模型和抽取出的骨架模型。由于点云模型是
三维的点的集合,而抽取出的骨架模型却是一个记录着树形结构的逻辑信息,它们无法进行直接的比较。本文采取的做法是将骨架的
树形逻辑信息用圆台和球进行堆叠从而将其转化为三维的表示。

具体的做法是对骨架中的每个节点,根据其半径构造出一个球体。然后对于每个父子关系,用一个圆台来表示其枝干,圆台的底半径
等于父节点的半径,圆台的顶半径等于子节点的半径。然后对于每个点云模型中的点,用数学公式判断其是否存在于骨架的三维表示中
的球体或圆台中,如果存在,则表示匹配成功,否则表示匹配失败。最后用成功点数与总点数的比值来表示骨架抽取的还原度。定义如下:\\
\begin{definition}
	\[ \mathbf{R_s}=\frac{S}{N} \]
\end{definition}

其中$S$表示匹配成功的点数,而$N$表示点云模型的总点数。$\mathbf{R_s}$的值区间为$[0,1]$。

注意,若用经过枝干合并轻量化处理的骨架进行骨架抽取还原度计算,其值必定会比直接从点云中抽取出来的模型要小,因为模型经过
简化后,与原点云模型的匹配度也必将降低。本文的目标只是尽可能在还原度降低不多的情况下,对骨架进行尽可能多的轻量化。

\section{建模还原度计算}
将图片序列信息量$\mathbf{I}$、三维重建还原度$\mathbf{R_{3d}}$和骨架抽取还原度$\mathbf{R_s}$代入建模还原度$\mathbb{Q}$
的定义式中,可以得到建模还原度的计算式:\\
\begin{equation}
	\mathbb{Q}= (1-\left[\frac{a}{b}\right]^n)\cdot \frac{1}{n}\sum_{i=1}^n \frac{P_i}{P_i+O_i} \cdot \frac{S}{N}
\end{equation}

\section{本章小节}
本章提出了基于图像树木轻量化建模的质量评价方法。首先提出了建模还原度的概念,它包含了三个子项:图像序列信息量、
三维重建还原度以及骨架抽取还原度。它们分别代表了图像序列对真实树木的信息携带量、点云模型和图像序列的匹配度、骨架模型与点云模型的匹配度。
本文对这三个子项的由来和计算方法都进行了阐述,并将它们融合给出了建模还原度的计算式。

\chapter{基于用户交互的模型改善}
通过自动化算法生成的模型难免含有算法设计者的主观想法在其中,比如树木何时应该分支,何时
应该合并等等。并且算法在处理实际的模型时,通常带有一定的二义性。未免过于主观地决定了模
型的最终成型,本文提倡在自动化生成模型以后,应该将模型
的建立延迟到应用程序,或者使用该模型的最终用户。这样可以使得本文所建立的一系列方法适用于
更广的情形。将自动化算法与人工交互联合使用,可以大大地提高最终模型与具体需求之间的耦合度。

本文根据此需求,开发完成了一个树木模型用户交互平台,可以用于对本文的基于图像所建立的树木
模型进行进一步地编辑和完善。该平台集成了几种不同轻量化级别的模型表示,模型的编辑,算法的
演示等功能,大大的提高了算法验证的直观度和用户与应用级模型改善的方便性。该平台的具体功能
见用例图\ref{fig:usecase}。

\begin{figure}[H]
	\centering
	\includegraphics[height=6cm]{usecase.jpg}
	\caption{用户交互平台用例图}
	\label{fig:usecase}
\end{figure}

对于每个用例,本文用表格和图片对其进行进一步说明,由于本文并不将重点放在软件开发上,所以
在本文中未给出具体的系统分析和设计。本文试图对该用户交互平台的功能进行阐述,从而在功能性上
对自动化算法进行人工加强和改善。

\section{交互平台用例说明}
\clearpage
\subsection{加载树木点云文件}

\begin{figure}[H]
	\centering
	\includegraphics[height=6cm]{uc1.png}
	\caption{用例1图示:用开源软件MeshLab打开的树木点云文件}
	\label{fig:uc1}
\end{figure}

\begin{table}[H]
	\centering
\begin{tabular}{|l|p{8cm}|}
	\hline
	用例名称: & 加载树木点云文件\\
	\hline
	用例标志号: & 1\\
	\hline
	参与者: & 用户\\
	\hline
	简要说明: & 用户可以按键'l'来加载存储为".ply"开源格式的名为"Tree.ply"的点云文件,该
	后缀名的点云模型可以用开源软件Meshlab来打开观察,如图\ref{fig:uc1}\\
	\hline
	前置条件: & 树木的模型文件存在于./Models/文件夹下\\
	\hline
	基本事件流: & 1. 用户按下'l'键\\
	 & 2. 系统去./Models/文件夹下搜索名为"Tree.ply"的点云文件\\
	 & 3. 将点云文件加载入内存\\
	 & 4. 用例终止\\
	\hline
	异常事件流: & 1. 若点云文件不存在,则不会加载入内存\\
	\hline
	后置条件: & 无\\
	\hline
\end{tabular}
\end{table}

\clearpage
\subsection{显示骨架模型}
\begin{figure}[H]
	\centering
	\includegraphics[height=6cm]{uc2.png}
	\caption{用例图示:显示骨架模型}
	\label{fig:uc2}
\end{figure}

\begin{table}[H]
	\centering
\begin{tabular}{|l|p{8cm}|}
	\hline
	用例名称: & 显示骨架模型\\
	\hline
	用例标志号: & 2\\
	\hline
	参与者: & 用户\\
	\hline
	简要说明: & 用户可以通过选择显示骨架模型来观察树木的骨架,具体效果见图\ref{fig:uc2}\\
	\hline
	前置条件: & 树木的模型文件已经被加载进入软件平台\\
	\hline
	基本事件流: & 1. 用户按下'm'键将显示模式切换到骨架模式\\
	 & 2. 将树木骨架模型用冯氏光照模型渲染至视口\\
	 & 3. 用例终止\\
	\hline
	异常事件流: & 1. 若模型文件未加载,则不会显示对应模型骨架\\
	\hline
	后置条件: & 无\\
	\hline
\end{tabular}
\end{table}

\clearpage
\subsection{显示点云模型}
\begin{figure}[H]
	\centering
	\includegraphics[height=6cm]{uc3.png}
	\caption{用例图示:显示点云模型}
	\label{fig:uc3}
\end{figure}

\begin{table}[H]
	\centering
\begin{tabular}{|l|p{8cm}|}
	\hline
	用例名称: & 显示点云模型\\
	\hline
	用例标志号: & 3\\
	\hline
	参与者: & 用户\\
	\hline
	简要说明: & 用户可以通过选择显示点云模型来观察树木的点云表示,具体效果见图\ref{fig:uc3}\\
	\hline
	前置条件: & 树木的模型文件已经被加载进入软件平台\\
	\hline
	基本事件流: & 1. 用户按下'm'键切换显示模式到点云模式\\
	 & 2. 将点云模型用平滑光照模型渲染至视口\\
	 & 3. 用例终止\\
	\hline
	异常事件流: & 1. 若模型文件未加载,则不会显示对应点云模型\\
	\hline
	后置条件: & 无\\
	\hline
\end{tabular}
\end{table}

\clearpage
\subsection{显示体素模型}
\begin{figure}[H]
	\centering
	\includegraphics[height=6cm]{uc4.png}
	\caption{用例图示:显示体素模型}
	\label{fig:uc4}
\end{figure}

\begin{table}[H]
	\centering
\begin{tabular}{|l|p{8cm}|}
	\hline
	用例名称: & 显示体素模型\\
	\hline
	用例标志号: & 4\\
	\hline
	参与者: & 用户\\
	\hline
	简要说明: & 用户可以通过选择显示体素模型来观察树木的体素表示,具体效果见图\ref{fig:uc4}\\
	\hline
	前置条件: & 树木的模型已经被加载进入软件平台\\
	\hline
	基本事件流: & 1. 用户按下'v'键\\
	 & 2. 将体素模型用冯氏光照模型渲染至视口\\
	 & 3. 用例终止\\
	\hline
	异常事件流: & 1. 若模型文件不存在,则不会显示对应点云模型\\
	\hline
	后置条件: & 无\\
	\hline
\end{tabular}
\end{table}

\clearpage
\subsection{节点编辑}
\begin{figure}[H]
	\centering
	\subfloat[节点插入(前)]{\includegraphics[width=0.26\linewidth]{uc5_ins1.png}}\hfill
	\subfloat[节点删除(前)]{\includegraphics[width=0.26\linewidth]{uc5_rm1.png}}\hfill
	\subfloat[节点移动(前)]{\includegraphics[width=0.26\linewidth]{uc5_mv1.png}}\hfill
	\subfloat[节点插入(后)]{\includegraphics[width=0.26\linewidth]{uc5_ins2.png}}\hfill
	\subfloat[节点删除(后)]{\includegraphics[width=0.26\linewidth]{uc5_rm2.png}}\hfill
	\subfloat[节点移动(后)]{\includegraphics[width=0.26\linewidth]{uc5_mv2.png}}\hfill
	\caption{用例图示:节点编辑}
	\label{fig:uc5}
\end{figure}

\begin{table}[H]
	\centering
\begin{tabular}{|l|p{10cm}|}
	\hline
	用例名称: & 节点编辑\\
	\hline
	用例标志号: & 5\\
	\hline
	参与者: & 用户\\
	\hline
	简要说明: & 可以对选中的节点进行丰富的编辑操作,具体效果见图\ref{fig:uc5}。
	其中橙色球体表示当前选中节点,半透明蓝色部分表示当前节点其下的子树,褐色区域表示树的其他部分\\
	\hline
	前置条件: & 树木的显示模式为骨架模型\\
	\hline
	基本事件流: & 1. 用户按下'g'键\\
	 & 2. 树木骨架从当前节点按用户视角右方向长出一个子节点\\
	 & 3. 调整当前节点为新增的子节点\\
	\hline
	可选事件流1: & 1. 用户按下'p'键\\
	 & 2. 树木删除当前节点以下的子树\\
	\hline
	可选事件流2: & 1. 用户按下鼠标左键拖动\\
	 & 2. 当前节点及其以下的子树在当前平面内平移\\
	\hline
	可选事件流3: & 1. 用户按下'n','f'键\\
	 & 2. 当前节点及其以下的子树在z平面内外平移\\
	\hline
\end{tabular}
\end{table}

\clearpage
\subsection{简化模型}
\begin{figure}[H]
	\centering
	\subfloat[模型简化(前)]{\includegraphics[width=0.4\linewidth]{uc6_1.png}}\hfill
	\subfloat[模型简化(后)]{\includegraphics[width=0.4\linewidth]{uc6_2.png}}\hfill
	\caption{用例图示:简化模型}
	\label{fig:uc6}
\end{figure}

\begin{table}[H]
	\centering
\begin{tabular}{|l|p{8cm}|}
	\hline
	用例名称: & 简化模型\\
	\hline
	用例标志号: & 6\\
	\hline
	参与者: & 用户\\
	\hline
	简要说明: & 用户可以通过选择某棵子树,对其进行如第\ref{cha:branchcombine}章中的合并操作,以简化模型。具体效果见图\ref{fig:uc6}\\
	\hline
	前置条件: & 树木的显示模式为骨架模型\\
	\hline
	基本事件流: & 1. 用户按下's'键\\
	 & 2. 对当前节点及其表示的子树进行基于合并的简化\\
	 & 3. 用例终止\\
	\hline
	后置条件: & 无\\
	\hline
\end{tabular}
\end{table}

\clearpage
\subsection{泛洪算法演示}
\begin{figure}[H]
	\centering
	\subfloat[输入:从点云中得到的体素模型]{\includegraphics[width=0.26\linewidth]{uc7_1.png}}\hfill
	\subfloat[第1次迭代]{\includegraphics[width=0.26\linewidth]{uc7_2.png}}\hfill
	\subfloat[第4次迭代]{\includegraphics[width=0.26\linewidth]{uc7_3.png}}\hfill
	\subfloat[第7次迭代]{\includegraphics[width=0.26\linewidth]{uc7_4.png}}\hfill
	\subfloat[第10次迭代]{\includegraphics[width=0.26\linewidth]{uc7_5.png}}\hfill
	\subfloat[泛洪算法完成]{\includegraphics[width=0.26\linewidth]{uc7_6.png}}
	\caption{用例图示:泛洪算法演示}
	\label{fig:uc7}
\end{figure}

\begin{table}[H]
	\centering
\begin{tabular}{|l|p{8cm}|}
	\hline
	用例名称: & 泛洪算法演示\\
	\hline
	用例标志号: & 7\\
	\hline
	参与者: & 用户\\
	\hline
	简要说明: & 用户可以根据输入的点云,进行泛洪算法的可视化单步观察。具体效果见图\ref{fig:uc7}\\
	\hline
	前置条件: & 树木的点云数据已经被加载进入程序,并且打开了体素模型模式\\
	\hline
	基本事件流: & 1. 用户按下'e'键\\
	 & 2. 对当前节点进行泛洪并找到其字节点\\
	 & 3. 选中其子节点,并重复1和2中的步骤\\
	 & 4. 当达到叶子节点时,用例结束\\
	\hline
	后置条件: & 无\\
	\hline
\end{tabular}
\end{table}

\clearpage
\section{利用用户交互提升算法完整性}
根据用户交互平台提供的一系列功能,可以对本文提出的骨架抽取算法进行进一步的完善。由于依靠自动化
骨架抽取算法不可能100\%的恢复出树木的骨架信息,它可能带有一定的骨架缺失,因此有必要依靠少量的人工
交互与指引,进一步完善算法中的缺失部分,从而得到一个完整的骨架模型。这里的人工指引并不是指的逐
个节点的手工编辑,而是对某些缺失的骨架给出一个方向的指引,使得其在缺失方向上重新进行泛洪,而自动化
地补全缺失的信息。这种方式比起手工对缺失的骨架进行建模效率高出许多,因为用户只要给出一个缺失的方向
即可。

图\ref{fig:intalg}展示了通过用户交互的方法指引泛洪算法的进一步完善的过程。
如图\ref{fig:intalg}(a),该树木模型已经经过了本文前文所介绍的骨架抽取步骤。然而与点云模型相对比,
发现当前节点处仍有部分的枝干丢失。经过分析,该枝干丢失可能的原因主要是骨架抽取算法的分支判定条件
并没有将当前分支判定进去。一种可行的办法是调整分支条件的参数,但是这对于用户来说并不是一种很好
的解决办法,因为在实际应用中,要用户去了解一个算法内部的参数是不大可能的。因此本文提出了基于用户
交互与指引的方法,用户只需要从缺失枝干的父枝干出发,引出一个子节点,而泛洪算法将由该子节点开始
继续探索该缺失部分。图\ref{fig:intalg}(b)给出了一个用户指引生成的节点,因此泛洪算法将继续从该
子节点开始,恢复该缺失枝条的骨架。最终的恢复结果如图\ref{fig:intalg}(c)所示。其中橙色球体代表当前
节点,红色区域代表点云,褐色区域代表已抽取骨架,蓝色区域代表通过用户交互恢复出来的枝条。

\begin{figure}[H]
	\centering
	\subfloat[缺失的枝条]{\includegraphics[width=0.3\linewidth]{before.png}}\hfill
	\subfloat[新插入节点作为算法指引]{\includegraphics[width=0.3\linewidth]{direction.png}}\hfill
	\subfloat[恢复的枝条]{\includegraphics[width=0.3\linewidth]{after.png}}
	\caption{基于用户交互提升算法完整性}
	\label{fig:intalg}
\end{figure}

可见,基于用户交互对算法进行快捷而方便的指引,相较于让用户逐个节点去进行完善,这种方法对于缺失枝干
的恢复具有重要意义。首先,该方法只用引出一个节点作为新的种子点,泛洪算法就可以递归性地恢复整个枝干,
这比起让用户去逐个节点的编辑,大大的节省了建模的时间;同时,由于运用本文介绍的骨架抽取算法所抽取出
的骨架已经大部分的还原了原始骨架,因此需要的用户交互只是及其少量的,这就保证了建模还是由自动化的方法
主导的,有着很好的便捷性;最后,若用户对于最终恢复的模型还有局部需要微调,则可以配合前面的模型编辑
功能进行编辑,这种编辑是建立在已经成型的树木骨架的基础上的,因此比起从头开始建模,也是更加方便和
快捷的方式。

\section{本章小节}
本章给出了一个基于用户交互的模型改善平台,它可以解决自动化算法所带来的一些主观性和二义性问题,通过
最终用户或应用来对自动化生成的模型进行改善,可以使本文给出的基于图像的树木轻量化算法普及到更多的应用
和需求。该平台集成了加载树木点云文件、显示骨架模型、显示点云模型、显示体素模型、节点编辑、简化模型、
以及算法演示的功能,能够很方便和快捷地对自动化生成的模型进行观察和加工。本章还给出了各个功能具体的
用例图和表格加以说明。


\chapter{实验过程与分析}

\section{实验环境}
对于前面提出的基于图像的树木轻量化建模方法,本文做了大量实验验证了其可行性。
本文使用的拍摄工具是高分辨率安卓手机索尼LT26ii,该手机最高分辨率能达到4000x3000,
已经足够实验的需求。拍摄地点为同济大学嘉定校区以及某住宅小区。本文实验所使用计算机
操作系统平台为XUbuntu 12.04,CPU为双核Intel(R) Core(TM) i5-3230M @ 2.60GHz。显卡为
NVidia GeForce GT750M。 所有实验程序均使用C/C++语言完成,使用g++编译器进行编译,
同时使用OpenGL version 4.3图形硬件接口完成了第八章提及的可视化平台。

\section{实验结果与分析}
本文以三棵树的建模结果作为展示和分析的依据,其中第一棵树拍摄帧数为11帧,见图\ref{fig:sample1}。
第二棵树的拍摄帧数为12帧,见图\ref{fig:sample2}。第三棵树的拍摄帧数为20帧,见图\ref{fig:sample3}。

% sample 1
\begin{figure}[H]
	\captionsetup[subfigure]{labelformat=empty}
	\subfloat[]{
	\includegraphics[height=2cm]{sample1_1.jpg}}
	\hspace{1mm}
	\subfloat[]{
	\includegraphics[height=2cm]{sample1_2.jpg}}
	\hspace{1mm}
	\subfloat[]{
	\includegraphics[height=2cm]{sample1_3.jpg}}
	\hspace{1mm}
	\subfloat[]{
	\includegraphics[height=2cm]{sample1_4.jpg}}
	\hspace{1mm}
	\subfloat[]{
	\includegraphics[height=2cm]{sample1_5.jpg}}
	\hspace{1mm}
	\subfloat[]{
	\includegraphics[height=2cm]{sample1_6.jpg}}
	\hspace{1mm}
	\subfloat[]{
	\includegraphics[height=2cm]{sample1_7.jpg}}
	\hspace{1mm}
	\subfloat[]{
	\includegraphics[height=2cm]{sample1_8.jpg}}
	\hspace{1mm}
	\subfloat[]{
	\includegraphics[height=2cm]{sample1_9.jpg}}
	\hspace{1mm}
	\subfloat[]{
	\includegraphics[height=2cm]{sample1_10.jpg}}
	\hspace{1mm}
	\subfloat[]{
	\includegraphics[height=2cm]{sample1_11.jpg}}
	\caption{树木1图像序列,帧数为11帧,拍摄地点为某住宅小区}
	\label{fig:sample1}
\end{figure}

% sample 2
\begin{figure}[H]
	\captionsetup[subfigure]{labelformat=empty}
	\subfloat[]{
	\includegraphics[height=2cm]{sample2_1.jpg}}
	\hspace{1mm}
	\subfloat[]{
	\includegraphics[height=2cm]{sample2_2.jpg}}
	\hspace{1mm}
	\subfloat[]{
	\includegraphics[height=2cm]{sample2_3.jpg}}
	\hspace{1mm}
	\subfloat[]{
	\includegraphics[height=2cm]{sample2_4.jpg}}
	\hspace{1mm}
	\subfloat[]{
	\includegraphics[height=2cm]{sample2_5.jpg}}
	\hspace{1mm}
	\subfloat[]{
	\includegraphics[height=2cm]{sample2_6.jpg}}
	\hspace{1mm}
	\subfloat[]{
	\includegraphics[height=2cm]{sample2_7.jpg}}
	\hspace{1mm}
	\subfloat[]{
	\includegraphics[height=2cm]{sample2_8.jpg}}
	\hspace{1mm}
	\subfloat[]{
	\includegraphics[height=2cm]{sample2_9.jpg}}
	\hspace{1mm}
	\subfloat[]{
	\includegraphics[height=2cm]{sample2_10.jpg}}
	\hspace{1mm}
	\subfloat[]{
	\includegraphics[height=2cm]{sample2_11.jpg}}
	\hspace{1mm}
	\subfloat[]{
	\includegraphics[height=2cm]{sample2_12.jpg}}
	\caption{树木2图像序列,帧数为12帧,拍摄地点为同济大学嘉定校区}
	\label{fig:sample2}
\end{figure}


% sample 3
\begin{figure}[H]
	\captionsetup[subfigure]{labelformat=empty}
	\subfloat[]{
	\includegraphics[height=2cm]{sample3_1.jpg}}
	\hspace{1mm}
	\subfloat[]{
	\includegraphics[height=2cm]{sample3_2.jpg}}
	\hspace{1mm}
	\subfloat[]{
	\includegraphics[height=2cm]{sample3_3.jpg}}
	\hspace{1mm}
	\subfloat[]{
	\includegraphics[height=2cm]{sample3_4.jpg}}
	\hspace{1mm}
	\subfloat[]{
	\includegraphics[height=2cm]{sample3_5.jpg}}
	\hspace{1mm}
	\subfloat[]{
	\includegraphics[height=2cm]{sample3_6.jpg}}
	\hspace{1mm}
	\subfloat[]{
	\includegraphics[height=2cm]{sample3_7.jpg}}
	\hspace{1mm}
	\subfloat[]{
	\includegraphics[height=2cm]{sample3_8.jpg}}
	\hspace{1mm}
	\subfloat[]{
	\includegraphics[height=2cm]{sample3_9.jpg}}
	\hspace{1mm}
	\subfloat[]{
	\includegraphics[height=2cm]{sample3_10.jpg}}
	\hspace{1mm}
	\subfloat[]{
	\includegraphics[height=2cm]{sample3_11.jpg}}
	\hspace{1mm}
	\subfloat[]{
	\includegraphics[height=2cm]{sample3_12.jpg}}
	\hspace{1mm}
	\subfloat[]{
	\includegraphics[height=2cm]{sample3_13.jpg}}
	\hspace{1mm}
	\subfloat[]{
	\includegraphics[height=2cm]{sample3_14.jpg}}
	\hspace{1mm}
	\subfloat[]{
	\includegraphics[height=2cm]{sample3_15.jpg}}
	\hspace{1mm}
	\subfloat[]{
	\includegraphics[height=2cm]{sample3_16.jpg}}
	\hspace{1mm}
	\subfloat[]{
	\includegraphics[height=2cm]{sample3_17.jpg}}
	\hspace{1mm}
	\subfloat[]{
	\includegraphics[height=2cm]{sample3_18.jpg}}
	\hspace{1mm}
	\subfloat[]{
	\includegraphics[height=2cm]{sample3_19.jpg}}
	\hspace{1mm}
	\subfloat[]{
	\includegraphics[height=2cm]{sample3_20.jpg}}
	\hspace{1mm}
	\caption{树木3图像序列,帧数为20帧,拍摄地点为同济大学嘉定校区}
	\label{fig:sample3}
\end{figure}

本文还给出了每个图片序列三维重建出的点云模型、从点云模型中直接抽取的骨架模型以及
经过枝干合并方法轻量化得到的骨架模型。为了从直观上比较重建模型和图片序列的相似度,
本文从正面和侧面对所得模型进行投影,以方便从二维的视角判断其相似程度。图\ref{fig:s1proj1}和
图\ref{fig:s1proj2}分别给出了树木样本1在正面投影和侧面投影的比对情况。图\ref{fig:s2proj1}和
图\ref{fig:s2proj2}分别给出了树木样本2在正面投影和侧面投影的比对情况。图\ref{fig:s3proj1}和
图\ref{fig:s3proj2}分别给出了树木样本3在正面投影和侧面投影的比对情况。
\begin{figure}[H]
	\centering
	\subfloat[树木正面图像]{\includegraphics[width=6cm]{s1img1.jpg}}\hspace{3em}
	\subfloat[树木点云模型]{\includegraphics[width=6cm]{s1point1.png}}\hspace{3em}
	\subfloat[树木骨架模型]{\includegraphics[width=6cm]{s1skl1.png}}\hspace{3em}
	\subfloat[树木轻量化骨架模型]{\includegraphics[width=6cm]{s1lw1.png}}\hspace{3em}
	\caption{树木样本1正面投影比对}
	\label{fig:s1proj1}
\end{figure}
\begin{figure}
	\centering
	\subfloat[树木侧面图像]{\includegraphics[width=6cm]{s1img2.jpg}}\hspace{3em}
	\subfloat[树木点云模型]{\includegraphics[width=6cm]{s1point2.png}}\hspace{3em}
	\subfloat[树木骨架模型]{\includegraphics[width=6cm]{s1skl2.png}}\hspace{3em}
	\subfloat[树木轻量化骨架模型]{\includegraphics[width=6cm]{s1lw2.png}}\hspace{3em}
	\caption{树木样本1侧面投影比对}
	\label{fig:s1proj2}
\end{figure}
\begin{figure}
	\centering
	\subfloat[树木正面图像]{\includegraphics[width=6cm]{s2img1.jpg}}\hspace{3em}
	\subfloat[树木点云模型]{\includegraphics[width=6cm]{s2point1.png}}\hspace{3em}
	\subfloat[树木骨架模型]{\includegraphics[width=6cm]{s2skl1.png}}\hspace{3em}
	\subfloat[树木轻量化骨架模型]{\includegraphics[width=6cm]{s2lw1.png}}\hspace{3em}
	\caption{树木样本2正面投影比对}
	\label{fig:s2proj1}
\end{figure}
\begin{figure}
	\centering
	\subfloat[树木侧面图像]{\includegraphics[width=6cm]{s2img2.jpg}}\hspace{3em}
	\subfloat[树木点云模型]{\includegraphics[width=6cm]{s2point2.png}}\hspace{3em}
	\subfloat[树木骨架模型]{\includegraphics[width=6cm]{s2skl2.png}}\hspace{3em}
	\subfloat[树木轻量化骨架模型]{\includegraphics[width=6cm]{s2lw2.png}}\hspace{3em}
	\caption{树木样本2侧面投影比对}
	\label{fig:s2proj2}
\end{figure}
\begin{figure}
	\centering
	\subfloat[树木正面图像]{\includegraphics[width=6cm]{s3img1.jpg}}\hspace{3em}
	\subfloat[树木点云模型]{\includegraphics[width=6cm]{s3point1.png}}\hspace{3em}
	\subfloat[树木骨架模型]{\includegraphics[width=6cm]{s3skl1.png}}\hspace{3em}
	\subfloat[树木轻量化骨架模型]{\includegraphics[width=6cm]{s3lw1.png}}\hspace{3em}
	\caption{树木样本3正面投影比对}
	\label{fig:s3proj1}
\end{figure}
\begin{figure}
	\centering
	\subfloat[树木侧面图像]{\includegraphics[width=6cm]{s3img2.jpg}}\hspace{3em}
	\subfloat[树木点云模型]{\includegraphics[width=6cm]{s3point2.png}}\hspace{3em}
	\subfloat[树木骨架模型]{\includegraphics[width=6cm]{s3skl2.png}}\hspace{3em}
	\subfloat[树木轻量化骨架模型]{\includegraphics[width=6cm]{s3lw2.png}}\hspace{3em}
	\caption{树木样本3侧面投影比对}
	\label{fig:s3proj2}
\end{figure}

\clearpage
可以看出,无论从正面还是侧面,通过本文轻量化建模方法所得到的树木模型都具有很高的还原
性。由于树木1和树木的细小枝干比较多,所以对于其细节部分的还原度还有待提升。但是类似
树木3这种不含太多细小枝干的树木,本文可以给出很高的还原度。从另一方面来看,由于本文
的方法是轻量化建模,对于细小的枝干的舍弃也是无法避免的,所以从总体来看本文的方法对于
真实树木的轻量化建模效果还是十分客观的。

表\ref{tab:restore}分别对图片序列信息量、三维重建还原度、骨架抽取还原度以及总的还原度给出了量化
的实验计算结果。注意,图片序列信息量的计算,本文根据实验所得的最佳$a/b=0.8$的结果来进行计算。
从表中可以看出,树木样本3的还原度最高,而树木样本2的还原度最低。分别比较它们的三维重建还原
度和骨架抽取还原度可以看出,图片数量的越多,样本结构越简单,则三维还原度的还原度就越高。
而对于树木结构复杂的树木,骨架抽取是一个难点,因为树木分支结构的复杂性容易引起骨架抽取的
二义性从而导致抽取不准确,所以对于样本3这样结构简单的树木来说,骨架抽取还原度最高,而对于
结构最复杂,细枝最多的样本2来说,骨架抽取还原度最低。
\\
\begin{table}[H]
	\caption{树木样本还原度统计}
	\centering
\begin{tabular}{c|cccc} \label{tab:restore}
	树木样本& 图片序列信息量 & 三维重建还原度 & 骨架抽取还原度 & 总还原度\\
	\hline
	1		& 0.988			 & 0.892		  & 0.871		   & 0.767\\
	2		& 0.931			 & 0.849		  & 0.852		   & 0.673\\
	3		& 0.988			 & 0.935		  & 0.953		   & 0.881\\
\end{tabular}
\end{table}

表\ref{tab:lw}从轻量化的角度,对比了三个树木样本从三维重建得到的点云数据文件的体积、
骨架抽取后得到的文件体积和经过轻量化以后得到的文件体积。从表中可以看出,三个样本从
点云数据到骨架数据体积都大大的减小了,对于以KB为单位的骨架体积,已经可以普及到一般的
Web应用来。对于后一步轻量化的步骤,主要是为了满足更高的应用需求,可以看出,对于骨架
结构中细枝较多的样本1和样本2,其轻量化所减小的模型体积比例要大于结构简单的样本3,这是
因为模型的轻量化主要是建立在对树木结构进行简化,而本来就比较简单的结构,这种轻量化的
程度就会减弱。
\\
\begin{table}[H]
	\caption{树木建模各阶段文件体积对比}
	\centering
	\begin{tabular}{c|ccc} \label{tab:lw}
		树木样本	&	三维重建点云体积	& 原骨架体积	& 轻量化骨架体积\\
		\hline
		1			&   2.5M				& 50.3K			& 8.2K\\
		2			&	4.4M				& 103.1K		& 15.1K\\
		3			&	2.2M				& 31.5K			& 7.3K\\
	\end{tabular}
\end{table}

\section{本章小节}
本章首先交代了本文所进行实验的软硬件平台等实验环境。然后给出了3个样本的实验结果,
从模型正面和侧面进行投影并从直观上比较了重建模型与输入图像序列的相似度。然后进一步运用
前一章给出的建模还原度的计算式,对三个样本进行了量化的计算,并分析了影响建模还原度
的一些因素。最后给出了3个样本重建模型的轻量化数据,并基于该分析了影响树木模型轻量化
的一些因素。从最后的实验结果,我们可以看出,本文提出的基于图像的轻量化建模方法具有
可行性和十分客观的重建效果。



%%% 其它部分
\backmatter

\makeatother

% 致谢
\cleardoublepage

%%% Local Variables:
%%% mode: latex
%%% TeX-master: "../main"
%%% End:

\begin{ack}
  衷心感谢我的导师贾金原教授。在这一年的毕业设计过程中,贾老师严谨的治学作风、
  渊博的学术造诣以及悉心的指导给了我莫大的帮助和激励。不仅使我学到了如何高效率
  地工作,也让我学到了在治学方面所应该持有的执着又平淡的态度。使我从一个对
  计算机图形方面一无所知的门外汉,成为了深入其中并乐此不疲的钻研者。在论文完成
  之际,对贾老师表示最衷心的谢意。

  感谢同济大学软件学院在这本科四年中对我的培养,让我学会了做人,也让我对软件开发
  产生了浓厚的兴趣。感谢同济大学,你终将是我引以为傲的母校。在此希望学校和学院
  都越来越好。
  
  感谢实验室全体同学对我的帮助和支持。本课题承蒙国家自然科学基金资助,特此致谢。

  感谢\LaTeX{}的存在,虽然它没有所见即所得的直观,但却为我排版省下了不少时间,也
  使我的论文格式漂亮了很多。

\end{ack}


% 参考文献
\cleardoublepage
\bibliographystyle{tongjibib}
\bibliography{ref/refs}


% 附录
%\cleardoublepage
%\begin{appendix}
%\input{data/appendix01}
%\end{appendix}

% 个人简历
%\cleardoublepage
%\include{data/resume}

\end{document}
